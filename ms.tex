% Copyright (c) 2025 Carl Martin Ludvig Sinander.

% This program is free software: you can redistribute it and/or modify
% it under the terms of the GNU General Public License as published by
% the Free Software Foundation, either version 3 of the License, or
% (at your option) any later version.

% This program is distributed in the hope that it will be useful,
% but WITHOUT ANY WARRANTY; without even the implied warranty of
% MERCHANTABILITY or FITNESS FOR A PARTICULAR PURPOSE. See the
% GNU General Public License for more details.

% You should have received a copy of the GNU General Public License
% along with this program. If not, see <https://www.gnu.org/licenses/>.

%                                   _     _
%    _ __  _ __ ___  __ _ _ __ ___ | |__ | | ___
%   | '_ \| '__/ _ \/ _` | '_ ` _ \| '_ \| |/ _ \
%   | |_) | | |  __/ (_| | |  | | | | |_) | |  __/
%   | .__/|_|  \___|\__,_|_| |_| |_|_.__/|_|\___|
%   |_|


%%% bug catcher
\RequirePackage[l2tabu,orthodox]{nag}

%%% document class
\documentclass[11pt,letterpaper,reqno,oneside]{book}

%%% settings
\input{preamble.tex}

%%% for \widthof
\usepackage{calc}

%%% break DOIs in bibliography
\setcounter{biburlnumpenalty}{100}

%%% raise hypertargets above baseline
\makeatletter
	\newcommand{\hyperdest}[1]{\Hy@raisedlink{\hypertarget{#1}{}}}
\makeatother

%%% custom commands
\DeclareMathOperator*{\cotwo}{\underline{co}}
\DeclareMathOperator*{\inftwo}{\underline{inf}}

% watermark on every page (for version control)
% \usepackage[anchor=ll,pos={0.1cm,0.5cm},fontsize=0.2cm,angle=0,alignment=l]{draftwatermark}
% \SetWatermarkText{\normalfont\hspace{0.37em}version:\\\normalfont {\datestyle\today}\\\normalfont\hspace{0.37em}at {\currenttime}}


%%% bibliography
\addbibresource{bibl.bib}


%______________________________________________________________________________




%    _____ _ _   _
%   |_   _(_) |_| | ___
%     | | | | __| |/ _ \
%     | | | | |_| |  __/
%     |_| |_|\__|_|\___|


\title{\scshape Uncertainty}

\author{Ludvig Sinander \\
University of Oxford}

\date{\small This version: 25 November 2025}

% \date{\emph{version:} {\datestyle\today} at {\currenttime} \\ \hspace{0pt} \\ \hspace{0pt} \\ \bfseries please report typos!}

\makeatletter
	\AtBeginDocument{ \hypersetup{
		pdftitle = {Uncertainty},
		pdfauthor = {Ludvig Sinander}
		} }
\makeatother



%______________________________________________________________________________




%    ____                                        _
%   |  _ \  ___   ___ _   _ _ __ ___   ___ _ __ | |_
%   | | | |/ _ \ / __| | | | '_ ` _ \ / _ \ '_ \| __|
%   | |_| | (_) | (__| |_| | | | | | |  __/ | | | |_
%   |____/ \___/ \___|\__,_|_| |_| |_|\___|_| |_|\__|


\begin{document}

\maketitle

\pagebreak
\hspace{1pt}\vfill
\noindent
Copyright \copyright{} 2025 Carl Martin Ludvig Sinander.

\begin{quotation}
\noindent
Permission is granted to copy, distribute and/or modify this document under the terms of the \href{https://www.gnu.org/licenses/fdl}{GNU Free Documentation License}, Version 1.3 or any later version published by the Free Software Foundation; with no Invariant Sections, no Front-Cover Texts, and no Back-Cover Texts. A copy of the license is included in the section entitled `GNU
Free Documentation License'.
\end{quotation}

\noindent
This is a `copyleft' licence.
Visit \href{https://www.gnu.org/licenses/copyleft}{gnu.org/licenses/copyleft} to learn more.



%%%%%%%%%%%%%%%%%%%%%%
%%%%%%%%%%%%%%%%%%%%%%
%%%%%%%%%%%%%%%%%%%%%%
\chapter*{Preface}
\label{preface}
%%%%%%%%%%%%%%%%%%%%%%
%%%%%%%%%%%%%%%%%%%%%%
%%%%%%%%%%%%%%%%%%%%%%

These are the notes for the uncertainty component of the first-year graduate microeconomics sequence at Oxford, which I first taught in autumn 2025. A few parts of these notes are drawn pretty much verbatim from \textcite{oo}; thanks to my co-authors acquiescing to this. Thanks to Malayvardhan Prajapati for expert proofreading, and to Oscar Calvert, Pakorn Nunta-aree, Augustus Smith and Kiisa Uusitalo for reporting typos.



%%%%%%%%%%%%%%%%%%%
%%%%%%%%%%%%%%%%%%%
% Table of contents
\pagebreak
\microtypesetup{protrusion=false}
\setcounter{tocdepth}{1}
\tableofcontents
\microtypesetup{protrusion=true}
%%%%%%%%%%%%%%%%%%%
%%%%%%%%%%%%%%%%%%%



\setcounter{chapter}{-1}
%%%%%%%%%%%%%%%%%%%%%%
%%%%%%%%%%%%%%%%%%%%%%
%%%%%%%%%%%%%%%%%%%%%%
\chapter{Introduction}
\label{ch0}
%%%%%%%%%%%%%%%%%%%%%%
%%%%%%%%%%%%%%%%%%%%%%
%%%%%%%%%%%%%%%%%%%%%%

These notes cover some of the essentials of choice and learning under uncertainty, at the graduate level. The focus is on economic questions. (Rather than on psychological experiments, whether in thought, lab, or field. Or functional analysis.) The topics are risk (e.g. expected utility, risk attitude, stochastic orders), ambiguity (in particular, subjective expected utility), and information (in especially its value). These topics are introduced in \cref{ch0:overview} below.

The mathematical prerequisites are relatively slight: some basic real analysis (limits, continuity, integrals) is presumed, as well as the separating hyperplane theorem. Some useful mathematical background is reviewed in \cref{math}.

Some basic choice theory is assumed, e.g. the meaning of terms like `transitive', `preference' and `utility representation'. We briefly review this background in \cref{ch0:review} below.

For supplementary/alternative reading, I particularly recommend \textcite{Kreps1988,Strzalecki2023} on risk and ambiguity (\cref{ch_risk,ch_mone,ch_ambi} of these notes) and \textcite{Liang2023} on information (\cref{ch_info} of these notes). Many other excellent texts cover aspects of risk and ambiguity, e.g. \textcite{MascolellWhinstonGreen1995,Gollier2001,Gilboa2009,Rubinstein2012,Kreps2013,Sarver2023}. For a deeper dive into risk, ambiguity, and choice more broadly, see \textcite{Fishburn1979}.



%%%%%%%%%%%%%%%%%%%%%%%%%%%%%%%%%%%
%%%%%%%%%%%%%%%%%%%%%%%%%%%%%%%%%%%
\section{Risk, ambiguity, and information}
\label{ch0:overview}
%%%%%%%%%%%%%%%%%%%%%%%%%%%%%%%%%%%
%%%%%%%%%%%%%%%%%%%%%%%%%%%%%%%%%%%

Most economic decisions are made under uncertainty. For example, when choosing between projects in an organisation, a manager can rarely predict with certainty what the return on each project would be, if chosen. Economists describe such uncertainty by drawing a conceptual distinction between \emph{alternatives} (also called `prizes', `consequences' or `outcomes') and \emph{prospects.}

An alternative is a complete description of whatever the decision-maker ultimately cares about (in the stylised economic model at hand). Often, each alternative is a real number, interpreted as a monetary amount; this captures a decision-maker who cares (only) about money, or more generally a decision-maker whose preferences are additively separable between money and everything else of interest to her (which can therefore be ignored for our purposes). In consumer-choice contexts, alternatives are bundles of commodities; in matching, alternatives are potential match partners; in public finance, alternatives might be income or wealth distributions.

\emph{Prospects} are the actual objects of choice, which the decision-maker must compare and ultimately choose between. Examples of prospects include projects, insurance policies, portfolios, spouses, occupations, and locations. Each prospect, if chosen, produces a certain alternative, and what the decision-maker ultimately cares about is not which prospect was chosen per se, but rather which alternative was produced.%
	\footnote{\label{footnote:consequentialism}This property of `caring only about which alternative was produced' is sometimes called \emph{consequentialism} by decision theorists. The way I've introduced things, consequentialism holds by definition of the alternatives: the whole point (to an economist) of introducing the concept of `alternatives' in the first place (as distinct from the prospects) is to guarantee consequentialism. Decision theorists often think of consequentialism more like an assumption; rather than letting the alternatives be whatever the decision-maker cares about as here, a decision theorist may fix the alternatives to be something or other (monetary prizes, say) and then assume, or not assume, that consequentialism is satisfied. From the perspective of economic modelling, this is conceptually confused and unhelpful. However, decision theorists are often in the business of \emph{psychological} modelling (studying behavioural quirks motivated by experiments), and for this it is arguably desirable to allow for conceptual confusion (`violating consequentialism' is one kind of quirk).}

The key point is that the map from prospects to alternatives it not known to the decision-maker. This may be due to the decision-maker's ignorance, or due to inherent randomness/uncertainty in which alternative will ultimately arise if a given prospect is chosen, or a combination of the two; since the decision-maker cannot distinguish between the two, this distinction will be irrelevant for our purposes.

The simple way of modelling the decision-maker's uncertainty about how prospects map into alternatives is to assume that each prospect is associated with a \emph{lottery,} meaning a probability distribution over alternatives (if prospect $p$ is chosen, then alternative $x$ arises with probability $p(x)$, alternative $y$ arises with probability $p(y)$, and so on), and to further assume that these probabilities are known both to the decision-maker and to us (the economic modellers).%
	\footnote{It is common to describe this assumption as meaning that the probabilities are `objective', but that is misleading: they can be `subjective' assessments by the decision-maker. What matters is, rather, that we (the economic modellers) know which lottery corresponds (in the decision-maker's opinion) to which prospect.}
In practice, we identify each prospect with its associated lottery, thereby reducing choice among prospects to choice among lotteries. When uncertainty is modelled in this way, it is conventionally called \emph{risk.} (We could equally well call it `quantifiable uncertainty'.) This formalism is the subject of \cref{ch_risk,ch_mone}. (The former chapter studies risk in general, while the latter focusses on the special case in which the alternatives are monetary.)

A more general way of modelling uncertainty eschews the assumptions of the risk formalism. Here we postulate (without any real loss of generality) a set of `states of the world', each of which is an exhaustive description of how all relevant events might turn out. Each prospect is associated with an \emph{act,} meaning a map from states to alternatives; the idea is that if this prospect is chosen, then which alternative is produced will depend on how events turn out (i.e. which state of the world is realised), and the act (mapping states to alternatives) specifies exactly \emph{how} it depends. To put it another way, knowing the `state of the world' amounts to knowing, for each prospect, which alternative it produces (by definition of the `state of the world'---recall they \emph{exhaustively} describe how all \emph{relevant} events turn out).

In practice, this more general modelling approach identifies each prospect with its associated act, reducing choice among prospects to choice among acts. This implicitly assumes that the decision-maker correctly understands which prospect corresponds to which act, of course, and that we (the economic modellers) also correctly understand this. It does not require any assumptions about agreement on probabilities, however; the decision-maker may or may not have a probabilistic belief (`prior') about the state of the world, and if she does then we (the economic modellers) need not know what it is.%
	\footnote{In the special case in which the decision-maker does have a probabilistic belief \emph{and} we as economic modellers know what it is, we recover the simple `risk' formalism from above; in this case, we and the decision-maker can each compute, for each act, the implied lottery (probability distribution over alternatives), and by definition of `alternative', this lottery is all that matters. (Here we are again leaning on consequentialism---see \cref{footnote:consequentialism} above.)}
When uncertainty is modelled in this way, it is conventionally called \emph{ambiguity} or \emph{`Knightian uncertainty'.}%
	\footnote{`Knightian' refers to \textcite{Knight1921}, though one could equally call it `Keynesian' after \textcite{Keynes1921}, who also distinguished between risk and ambiguity. It is not uncommon for decision theorists to refer to ambiguity simply as `uncertainty', but in these notes I use `uncertainty' in its ordinary English sense, rather than as a term of art.}
(Another name is `unquantifiable uncertainty'.)
We study this formalism in \cref{ch_ambi}.

One of the advantages of the more general `acts/ambiguity' formalism is that it permits us to talk about learning, i.e. the acquisition of information by the decision-maker about the unknown mapping from prospects to alternatives. In particular, we model learning as the decision-maker acquiring information about which state of the world prevails. She may acquire deterministic information (excluding certain states of the world, for example), but more generally she may observe the outcome of a noisy signal that is correlated with the state of the world, and thus provides statistical information about it. Such learning (to inform a choice between acts) is the subject of \cref{ch_info}.



%%%%%%%%%%%%%%%%%%%%%%%%%%%%%%%%%%%
%%%%%%%%%%%%%%%%%%%%%%%%%%%%%%%%%%%
\section{Review of preference and utility}
\label{ch0:review}
%%%%%%%%%%%%%%%%%%%%%%%%%%%%%%%%%%%
%%%%%%%%%%%%%%%%%%%%%%%%%%%%%%%%%%%

Let $A$ be a non-empty set. Recall that a \emph{binary relation} $\succsim$ on $A$ is formally a subset of $A \times A$, with `$a \succsim b$' (for $a,b \in A$) being shorthand for $(a,b) \in \mathord{\succsim}$. The \emph{strict part} of a binary relation $\succsim$ on $A$ is the binary relation $\succ$ on $A$ such that for all $a,b \in A$, $a \succ b$ holds if and only if $a \succsim b \not\succsim a$. The \emph{symmetric part} of a binary relation $\succsim$ on $A$ is the binary relation $\sim$ on $A$ such that for all $a,b \in A$, $a \sim b$ holds if and only if $a \succsim b \succsim a$.

A binary relation $\succsim$ on $A$ is called \emph{complete} iff for all $a,b \in A$, either $a \succsim b$ or $b \succsim a$, and is called \emph{transitive} iff for all $a,b,c \in A$, $a \succsim b \succsim c$ implies $a \succsim c$. For any $a,b \in A$, `$a \not\succsim b$' is shorthand for `it is not the case that $a \succsim b$', and `$a \precsim b$' means $b \succsim a$.

A \emph{preference} on $A$ is a complete and transitive binary relation on $A$. (It would perhaps be better to say `rational preference' or some such, since it is perfectly possible to imagine a decision-maker who is indecisive or exhibits cyclic choice patterns.) Given a preference $\succsim$ on $A$ and alternatives $a,b, \in A$, we interpret $a \succsim b$ as `the decision-maker weakly prefers $a$ to $b$', $a \succ b$ as `the decision-maker strictly prefers $a$ to $b$', and $a \sim b$ as `the decision-maker is indifferent between $a$ and $b$'.

The orthodox economist's interpretation of $\succsim$ is that it is not some psychological thing, but rather a description how a decision-maker actually \emph{chooses.}%
	\footnote{This is the party line. However, economists do frequently rely on `thicker' concepts of preference, often unwittingly. In utilitarian calculations, for example.}
We ought really to say `hypothetical' instead of `actual', since to learn the entire binary relation $\succsim$ from observing choices, one would need to observe the decision-maker's choice from every possible binary menu $\{a,b\} \subseteq A$, which is a lot of data except if $A$ is small.%
	\footnote{A further difficulty is that it is not obvious how indifference $a \sim b$ should be interpreted in choice terms.}

The `decision-maker' whose choices are captured by $\succsim$ may be an individual, but in economic models is it frequently an organisation, such as a firm or a committee.

\begin{exercise}
	%
	\label{exercise:condorcet}
	%
	Assume that $\abs*{A} \geq 3$. There is a committee comprised of three voters. Each voter $i \in \{1,2,3\}$ has a preference $\succsim_i$ with no indifferences (that is, $a \nsim_i b$ for all $a \neq b$ in $A$). Let $\succsim$ denote the majority relation: that is, for any alternatives $a,b \in A$, we have $a \succsim b$ if and only $\abs*{ 
	\{ i \in \{1,2,3\} : a \succ_i b \} } \geq 2$.

	\begin{enumerate}[label=(\alph*)]
	
		\item Prove or disprove: whatever the voters' preferences $(\mathord{\succsim_1},\mathord{\succsim_2},\mathord{\succsim_3})$, the majority relation $\succsim$ is complete.

		\item Prove or disprove: whatever the voters' preferences $(\mathord{\succsim_1},\mathord{\succsim_2},\mathord{\succsim_3})$, the majority relation $\succsim$ is transitive.
	
	\end{enumerate}
	%
\end{exercise}

\begin{definition}
	%
	\label{definition:utility}
	%
	Let $A$ be a non-empty set, let $\succsim$ be a binary relation on $A$. A function $U : A \to \R$ is said to \emph{represent} $\succsim$ if and only if for all $a,b \in A$, $a \succsim b$ iff $U(a) \geq U(b)$.
	%
\end{definition}

Such a function $U$ is called a \emph{utility function,} or \emph{utility representation of $\succsim$.} A decision-maker with preference $\succsim$ with utility representation $U$ behaves, when making decisions, as if she were maximising $U$: her choice from any non-empty menu $M \subseteq A$ will be (an element of) $\argmax_{a \in M} U(a)$. In principle, there is nothing that we (as analysts) can do with a utility function that cannot be done using the the `primitive' binary relation $\succsim$, but in practice maximising a real-valued function is mathematically more tractable than working with a binary relation. This tractability is the (only) reason why we typically work with utility functions.

\begin{exercise}
	%
	\label{exercise:utility}
	%
	Let $A$ be a non-empty set, and let $\succsim$ be a binary relation on $A$. Show that if $\succsim$ admits a utility representation, then $\succsim$ must be a preference.
	%
\end{exercise}

Intuitively, we can always move back and forth between preferences and utility functions. This intuition isn't actually quite right in general, but it is correct in simple cases:

\begin{proposition}
	%
	\label{prop:U_exist_countable}
	%
	Let $A$ be a non-empty set. If $A$ is finite or countable, then a binary relation on $A$ admits a utility representation iff it is a preference.
	%
\end{proposition}

The `only if' part of \Cref{prop:U_exist_countable} follows from \Cref{exercise:utility} above. For a proof of the `if' part, see \textcite[pp.~23--4]{Kreps1988}.

When the set $A$ is uncountable, a preference may admit no utility representation. This simply means that $\R$ is not large enough to represent the range of feelings that the decision-maker has about the (very many) alternatives in $A$. (Recall that by definition, a utility function carries $A$ into $\R$.) It is no more mysterious than that. You might find this slightly mind-bending because $\R$ is after all a very large set---but there are plenty of sets of yet greater cardinality (for example, the set of all functions $\R \to \R$). (The study of cardinality is a classic topic in set theory. Fun, but far removed from economics.)

The classic example of a preference which does not admit a utility representation is the lexicographic preference $\succsim$ on $A=\R^2$, which for any $a \equiv (a_1,a_2) \in A$ and $b \equiv (b_1,b_2) \in A$ satisfies $a \succsim b$ if and only if either (i)~$a_1 > b_1$ or (ii)~$a_1=b_1$ and $a_2>b_2$. To learn more (including a proof that the lexicographic preference admits no utility representation), see chapter~3 of \textcite{Kreps1988}.

In practice, the usual way of ensuring the existence of a utility representation is to impose topological assumptions; in particular, assuming that $A$ is a suitable topological space and that the preference $\succsim$ is continuous in a suitable sense. For example:

\begin{theorem}[\cite{Debreu1954}]
	%
	\label{theorem:debreu}
	%
	Let $A$ be a non-empty set, and $\succsim$ a preference on $A$. Suppose that $A$ is a convex subset of $\R^n$ (where $n \in \N$) and that for every $a \in A$, the upper and lower contour sets $\{ b \in A : b \precsim a \}$ and $\{ b \in A : b \succsim a \}$ are closed. Then $\succsim$ admits a utility representation.
	%
\end{theorem}

For a proof, see \textcite[chapter~2]{Rubinstein2012}.

Utility is \emph{ordinal,} in the sense that no information is lost if a strictly increasing transformation is applied to the utility function. Formally:

\begin{observation}
	%
	\label{observation:U_uniqueness}
	%
	Let $A$ be a non-empty set, and $\succsim$ a preference on $A$. Let $U$ and $V$ be functions $A \to \R$, and suppose that $U$ represents $\succsim$. Then $V$ represents $\succsim$ if and only if there exists a strictly increasing function $\phi : U(A) \to \R$ such that $V = \phi \circ U$.
	%
\end{observation}

\begin{exercise}
	%
	\label{exercise:pf_U_uniqueness}
	%
	Prove it!
	%
\end{exercise}



%%%%%%%%%%%%%%%%%%%%%%
%%%%%%%%%%%%%%%%%%%%%%
%%%%%%%%%%%%%%%%%%%%%%
\chapter{Affineness}
\label{ch_mix}
%%%%%%%%%%%%%%%%%%%%%%
%%%%%%%%%%%%%%%%%%%%%%
%%%%%%%%%%%%%%%%%%%%%%

% Copyright (c) 2025 Carl Martin Ludvig Sinander.

% This program is free software: you can redistribute it and/or modify
% it under the terms of the GNU General Public License as published by
% the Free Software Foundation, either version 3 of the License, or
% (at your option) any later version.

% This program is distributed in the hope that it will be useful,
% but WITHOUT ANY WARRANTY; without even the implied warranty of
% MERCHANTABILITY or FITNESS FOR A PARTICULAR PURPOSE. See the
% GNU General Public License for more details.

% You should have received a copy of the GNU General Public License
% along with this program. If not, see <https://www.gnu.org/licenses/>.

%%%%%%%%%%%%%%%%%%%%%%%%%%%%%%%%%%%%%%%%%%%%%%%%%%%%%%%%%%%%%%%%%%%%%%%

In this chapter, we introduce some mathematical concepts and results that will later (very easily) deliver the expected-utility representation theorems of \textcite{VonneumannMorgenstern1947} (in \cref{ch_risk}) and \textcite{AnscombeAumann1963} (in \cref{ch_ambi}). In particular, we introduce mixture spaces and affine functions on them, identify conditions under which an affine function admits an additive representation, and identify conditions for a preference to admit an affine utility representation.



%%%%%%%%%%%%%%%%%%%%%%%%%%%%%%%%%%%
%%%%%%%%%%%%%%%%%%%%%%%%%%%%%%%%%%%
\section{Mixture spaces}
\label{mix:def}
%%%%%%%%%%%%%%%%%%%%%%%%%%%%%%%%%%%
%%%%%%%%%%%%%%%%%%%%%%%%%%%%%%%%%%%

\begin{definition}[\cite{HersteinMilnor1953}]
	%
	\label{definition:mix}
	%
	A \emph{mixture space} is a set $\Pi$ equipped with an operation $(\pi,\alpha,\rho) \mapsto \pi_\alpha \rho$ that carries $\Pi \times [0,1] \times \Pi$ into $\Pi$ and satisfies, for all $\pi,\rho \in \Pi$,

	\begin{enumerate}[label=(\roman*)]
	
		\item \label{mix:id} $\pi_1 \rho = \pi$,

		\item \label{mix:sym} $\pi_\alpha \rho = \rho_{1-\alpha} \pi$ for every $\alpha \in [0,1]$, and

		\item \label{mix:comp} $(\pi_\alpha \rho)_\beta \rho = \pi_{\alpha\beta} \rho$ for all $\alpha,\beta \in [0,1]$.
	
	\end{enumerate}
	%
	(An `operation' is just a function with more concise notation; in particular, the function $\phi : \Pi \times [0,1] \times \Pi \to \Pi$ given by $\phi(\pi,\alpha,\rho) = \pi_\alpha \rho$ for all $\pi,\rho \in \Pi$ and $\alpha \in [0,1]$.)
	%
\end{definition}

Although the definition is abstract, the idea is always that $\pi_\alpha \rho \in \Pi$ is an $\alpha$-weighted `mixture' of $\pi$ and $\rho$, as the following examples illustrate.

\begin{exercise}
	%
	\label{exercise:mix_Rn}
	%
	Show that $\R_+^n$ (for $n \in \N$) is a mixture space when equipped with the operation given by $x_\alpha y \coloneqq \alpha x + (1-\alpha) y$ for all $x,y \in \R^n$ and $\alpha \in [0,1]$.
	%
\end{exercise}

\begin{exercise}
	%
	\label{exercise:mix_lottsimple}
	%
	Let $X$ be a non-empty set. We consider simple lotteries, meaning (probability mass functions of) finitely supported probability distributions over $X$. Formally, a \emph{simple lottery} is a function $p : X \to [0,1]$ such that $\supp(p) \coloneqq \{ x \in X : p(x) > 0 \}$ is finite and $\sum_{x \in \supp(p)} p(x) = 1$. Show that the set $\Delta^0(X)$ of all simple lotteries is a mixture space when equipped with the operation given by $p_\alpha q \coloneqq \alpha p + (1-\alpha) q$ for all $p,q \in \Delta^0(X)$ and $\alpha \in [0,1]$. (You can interpret this as a compound lottery: an $\alpha$-biased coin is flipped, and if it lands heads then an alternative is drawn from $p$, while if it lands tails then an alternative is instead drawn from $q$.)
	%
\end{exercise}

\begin{exercise}
	%
	\label{exercise:mix_acts}
	%
	Let $S$ and $X$ be finite sets, and let $\Delta(X)$ denote the set of all (probability mass functions of) lotteries over $X$; that is, $\Delta(X)$ is the set of all functions $p : X \to [0,1]$ such that $\sum_{x \in X} p(x)=1$. Show that the set $\Delta(X)^S$ of all maps $S \to \Delta(X)$ is a mixture space when equipped with the operation defined by $f_\alpha g \coloneqq \alpha f + (1-\alpha) g$ for all $f,g \in \Delta(X)^S$ and $\alpha \in [0,1]$. (You can interpret this as lottery compounding: given the state $s \in S$, an $\alpha$-biased coin is flipped. In case of heads, an alternative is drawn from the lottery $f(s)$, and in case of tails an alternative is instead drawn from $g(s)$.)
	%
\end{exercise}

\begin{exercise}
	%
	\label{exercise:mix_convex}
	%
	Show that if $\Pi$ is a convex subset of a real vector space equipped with the operation $(x,\alpha,y) \mapsto \alpha x + (1-\alpha) y$, then $\Pi$ is a mixture space. (This nests \Cref{exercise:mix_Rn,exercise:mix_lottsimple,exercise:mix_acts}.)
	%
\end{exercise}

\begin{exercise}
	%
	\label{exercise:mix_utility}
	%
	Fix $n \in \N$, and recall that a function $f : \R_+^n \to \R_+$ is called \emph{Cobb--Douglas} iff there exist constants $a_1,a_2,\dots,a_n \in [0,1]$ such that $a_1 + a_2 + \cdots + a_n = 1$ and, for every $x \in \R_+^n$, $f(x) = x_1^{a_1} x_2^{a_2} \cdots x_n^{a_n}$. Show that $\Pi$ is a mixture space when equipped with the operation $(f,\alpha,g) \mapsto f^\alpha g^{1-\alpha}$.
	%
\end{exercise}

\begin{exercise}
	%
	\label{exercise:mix_normals}
	%
	Let $\Pi$ be the set of all Normal PDFs on $\R$---that is, the set of all functions $\R \to \R$ given by
	%
	\begin{equation*}
		x \mapsto \left( 2\pi\sigma^2 \right)^{-1/2} \exp\left( -\frac{1}{2} \frac{(x-\mu)^2}{\sigma^2} \right)
	\end{equation*}
	%
	for some mean $\mu \in \R$ and variance $\sigma^2 \in (0,+\infty)$. Equip $\Pi$ with the operation whereby if $f_1,f_2 \in \Pi$ have respective means $\mu_1,\mu_2 \in \R$ and variances $\sigma_1^2,\sigma_2^2 \in (0,+\infty)$, then for any $\alpha \in [0,1]$, $(f_1)_\alpha (f_2)$ is the Normal PDF with mean $\alpha \mu_1 + (1-\alpha) \mu_2$ and variance $\alpha^2 (\sigma_1)^2 + (1-\alpha)^2 (\sigma_2)^2$.

	\begin{enumerate}[label=(\alph*)]
	
		\item Interpret this operation.

		\item Is $\Pi$ is a mixture space?
	
	\end{enumerate}
	%
\end{exercise}

Mixture spaces satisfy the following property, which we shall use later.

\begin{proposition}
	%
	\label{proposition:mix_twodistrib}
	%
	Let $\Pi$ be a mixture space. For any $\pi,\rho \in \Pi$ and any $\alpha,\beta,\gamma \in [0,1]$, $(\pi_\alpha \rho)_\gamma (\pi_\beta \rho) = \pi_{\gamma \alpha + (1-\gamma) \beta} \rho$.
	%
\end{proposition}

\begin{exercise}[hard]
	%
	\label{exercise:mix_twodistrib}
	%
	Prove it!
	%
\end{exercise}

\begin{exercise}[from \cite{Kreps1988}]
	%
	\label{exercise:mix_ident}
	%
	Let $\Pi$ be a mixture space. Prove that for any $\pi \in \Pi$ and any $\alpha \in [0,1]$, $\pi_\alpha \pi = \pi$.
	%
\end{exercise}

\begin{exercise}[inspired by Nemanja Antić]
	%
	\label{exercise:mix_nonprop}
	%
	Say that a mixture space $\Pi$ is \emph{determinate} iff for all $\pi,\rho,\sigma \in \Pi$ and $\alpha \in (0,1]$, $\pi_\alpha \sigma = \rho_\alpha \sigma$ implies $\pi = \rho$. Say that a mixture space $\Pi$ is \emph{associative} iff for all $\pi,\rho,\sigma \in \Pi$ and $\alpha,\beta \in [0,1]$, $(\pi_\alpha \rho)_\beta \sigma = \pi_{\alpha\beta} ( \rho_{\beta(1-\alpha)/(1-\alpha\beta)} \sigma )$.

	\begin{enumerate}[label=(\alph*)]

		\item Show that the mixture space in \Cref{exercise:mix_convex} is determinate and associative.
	
		\item Find an example of a mixture space $\Pi$ that is not determinate.

		\item Find an example of a mixture space $\Pi$ that is not associative.
	
	\end{enumerate}
	%
\end{exercise}



%%%%%%%%%%%%%%%%%%%%%%%%%%%%%%%%%%%
%%%%%%%%%%%%%%%%%%%%%%%%%%%%%%%%%%%
\section{Affineness and additive representations}
\label{mix:add}
%%%%%%%%%%%%%%%%%%%%%%%%%%%%%%%%%%%
%%%%%%%%%%%%%%%%%%%%%%%%%%%%%%%%%%%

\begin{definition}
	%
	\label{definition:affine}
	%
	Let $\Pi$ be a mixture space. A function $U : \Pi \to \R$ is called \emph{affine} if and only if $U(\pi_\alpha \rho) = \alpha U(\pi) + (1-\alpha) U(\rho)$ for all $\pi,\rho \in \Pi$ and $\alpha \in [0,1]$.
	%
\end{definition}

\begin{namedthm}[\Cref*{exercise:mix_Rn} {\normalfont (continued).}]
	%
	\label{exercise:mix_Rn_affine}
	%
	Fix a function $U : \R_+^n \to \R$.

	\begin{enumerate}[label=(\alph*)]
	
		\item Show that if there exists a vector $k \in \R^n$ and a constant $\beta \in \R$ such that $U(x) = k \cdot x + \beta$ for every $x \in \R_+^n$, then $U$ is affine.

		\item Prove the converse: if $U$ is affine, then there exists a vector $k \in \R^n$ and a constant $\beta \in \R$ such that $U(x) = k \cdot x + \beta$ for every $x \in \R_+^n$.
	
	\end{enumerate}
	%
\end{namedthm}

\begin{namedthm}[\Cref*{exercise:mix_lottsimple} {\normalfont (continued).}]
	%
	\label{exercise:mix_lottsimple_affine}
	%
	Fix a function $U : \Delta^0(X) \to \R$. For a simple lottery $p \in \Delta^0(X)$, `$\int u \dd p$' is shorthand for $\sum_{x \in \supp(p)} p(x) u(x)$.

	\begin{enumerate}[label=(\alph*)]
	
		\item Show that if there exists a function $u : X \to \R$ such that $U(p) \coloneqq \int u \dd p$ for each $p \in \Delta^0(X)$, then $U$ is affine.

		\item Prove the converse: if $U$ is affine, then there exists a function $u : X \to \R$ such that $U(p) \coloneqq \int u \dd p$ for each $p \in \Delta^0(X)$.
	
	\end{enumerate}
	%
\end{namedthm}

\begin{namedthm}[\Cref*{exercise:mix_acts} {\normalfont (continued).}]
	%
	\label{exercise:mix_acts_affine}
	%
	Fix a function $U : \Delta(X)^S \to \R$. For a lottery $p \in \Delta(X)$, `$\int u \dd p$' is shorthand for $\sum_{x \in X} p(x) u(x)$. (Recall that $X$ is finite.)

	\begin{enumerate}[label=(\alph*)]
	
		\item Show that if there exists a collection $(u_s)_{s \in S}$ of functions $X \to \R$ such that $U(f) = \sum_{s \in S} \int u_s \dd[f(s)]$ for each $f \in \Delta(X)^S$, then $U$ is affine.

		\item Prove the converse: if $U$ is affine, then there exists a collection $(u_s)_{s \in S}$ of functions $X \to \R$ such that $U(f) = \sum_{s \in S} \int u_s \dd[f(s)]$ for each $f \in \Delta(X)^S$.
	
	\end{enumerate}
	%
\end{namedthm}

\begin{exercise}
	%
	\label{exercise:mix_lottconts}
	%
	Show that the set of all CDFs $\R \to \R$ is a mixture space when equipped with the operation given by $(F,\alpha,G) \mapsto \alpha F + (1-\alpha) G$. Further show that for any bounded and measurable function $u : \R \to \R$, the map $F \mapsto \int u \dd F$ is affine.
	%
\end{exercise}

\begin{namedthm}[\Cref*{exercise:mix_utility} {\normalfont (continued).}]
	%
	\label{exercise:mix_utility_affine}
	%
	Find a non-constant affine map $\Pi \to \R$.
	%
\end{namedthm}



%%%%%%%%%%%%%%%%%%%%%%%%%%%%%%%%%%%
%%%%%%%%%%%%%%%%%%%%%%%%%%%%%%%%%%%
\section{The mixture-space theorem}
\label{mix:mix}
%%%%%%%%%%%%%%%%%%%%%%%%%%%%%%%%%%%
%%%%%%%%%%%%%%%%%%%%%%%%%%%%%%%%%%%

Our question in this section is what properties a preference $\succsim$ on a mixture space $\Pi$ must satisfy in order for $\succsim$ to admit an \emph{affine} utility representation $U$. Of course, if this is the case, then there are also many non-affine utility representations, since any strictly increasing transformation of a representation is itself a representation (\Cref{observation:U_uniqueness} in \cref{ch0}); we ask only that there exist an affine representation, not that there exist no non-affine ones.

\begin{definition}
	%
	\label{definition:mix_indep}
	%
	Let $\Pi$ be a mixture space. A preference $\succsim$ on $\Pi$ satisfies \emph{independence} iff for all $\pi,\rho,\sigma \in \Pi$ and $\alpha \in [0,1]$, $\pi \sim \rho$ implies $\pi_\alpha \sigma \sim \rho_\alpha \sigma$.
	%
\end{definition}

\begin{definition}
	%
	\label{definition:mix_continuity}
	%
	Let $\Pi$ be a mixture space. A preference $\succsim$ on $\Pi$ satisfies \emph{mixture continuity} iff for all $\pi,\rho,\sigma \in \Pi$ such that $\pi \succsim \rho \succsim \sigma$, the sets $\{ \alpha \in [0,1] : \pi_\alpha \sigma \succsim \rho \}$ and $\{ \alpha \in [0,1] : \pi_\alpha \sigma \precsim \rho \}$ are closed in $[0,1]$.
	%
\end{definition}

\begin{namedthm}[Mixture-space theorem {\normalfont (\cite{HersteinMilnor1953})}.]
	%
	\label{theorem:mix}
	%
	Let $\Pi$ be a mixture space, and let $\succsim$ be a preference on $\Pi$. There exists an affine function $U : \Pi \to \R$ that represents $\succsim$ if and only if $\succsim$ satisfies independence and mixture continuity. Furthermore, if two affine functions $U,V : \Pi \to \R$ both represent $\succsim$, then there exist $a>0$ and $b \in \R$ such that $U = a V + b$.
	%
\end{namedthm}

The second (`furthermore') claim asserts that affine utility representations are unique up to positive affine transformations. (A map $\R \to \R$ is called \emph{positive affine} iff it is both affine and strictly increasing.)

In the former claim, mixture continuity plays a merely technical role; the heavy lifting is done by independence, as will be clear from the proof. The \hyperref[theorem:mix]{mixture-space theorem} remains true if independence is modified in various ways, e.g. if it is weakened to `for all $\pi,\rho,\sigma \in \Pi$, $\pi \sim \rho$ implies $\pi_{1/2} \sigma \sim \rho_{1/2} \sigma$'.

\begin{exercise}
	%
	\label{exercise:mix_easy}
	%
	Prove the `only if' part of the first claim in the \hyperref[theorem:mix]{mixture-space theorem} (namely, that the existence of an affine representation implies independence and mixture continuity).
	%
\end{exercise}

\begin{proof}[Sketch proof of the {\hyperref[theorem:mix]{mixture-space theorem}}]
	%
	Let $\Pi$ be a mixture space, and let $\succsim$ be a preference on $\Pi$. Since this is a \emph{sketch} proof, we are allowed to add simplifying assumptions. So let's assume there are best and worst elements: that is, there exist $\overline{\pi},\underline{\pi} \in \Pi$ such that $\overline{\pi} \succsim \pi \succsim \underline{\pi}$ for every $\pi \in \Pi$. (There won't be any `sketchiness' apart from our imposition of this assumption.)

	Obviously if $\overline{\pi} \sim \underline{\pi}$, then among affine functions $U : \Pi \to \R$, all and only those that are constant represent $\succsim$, and obviously any two constant functions are positive affine transformations of each other. Assume for the remainder that $\overline{\pi} \succ \underline{\pi}$.

	We prove the second (`furthermore') claim last. For the first claim, the `only if' part was established in \Cref{exercise:mix_easy}, so it remains only to prove the `if' part. So suppose that $\succsim$ satisfies independence and mixture continuity; we must show that it admits an affine utility representation.

	The idea for the proof is this: for each $\pi \in \Pi$, we shall define $U(\pi)$ to be the unique $\alpha \in [0,1]$ such that $\pi \sim \overline{\pi}_\alpha \underline{\pi}$. The existence of such $\alpha$s comes from mixture continuity; uniqueness comes from independence. We will show that $U$ represents $\succsim$, again using independence. And we will show that this $U$ is affine, also by independence. That's it. Start with existence:

	\begin{namedthm}[Solvability claim.]
		%
		\label{claim:mix:solv}
		%
		For all $\pi,\rho,\sigma \in \Pi$ such that $\pi \succsim \rho \succsim \sigma$, there exists an $\alpha \in [0,1]$ such that $\rho \sim \pi_\alpha \sigma$.
		%
	\end{namedthm}

	\begin{proof}[Proof of the {\hyperref[claim:mix:solv]{solvability claim}}]
		%
		\renewcommand{\qedsymbol}{$\square$}
		We must show that the sets
		%
		\begin{equation*}
			B \coloneqq \{ \alpha \in [0,1] : \pi_\alpha \sigma \succsim \rho \}
			\quad \text{and} \quad
			W \coloneqq \{ \alpha \in [0,1] : \pi_\alpha \sigma \precsim \rho \} .
		\end{equation*}
		%
		are not disjoint. By inspection, $B$ and $W$ are both non-empty (why?), and satisfy $B \cup W = [0,1]$ (why?). Suppose toward a contradiction that $B$ and $W$ are disjoint, so that $B = [0,1] \setminus W$. Then $B \neq [0,1]$. Furthermore, since $W$ is closed (by mixture continuity), $B$ must be open in $[0,1]$. Finally, $B$ is closed by mixture continuity. To summarise, $B$ is clopen in $[0,1]$ (both open and closed) and satisfies $\varnothing \neq B \neq [0,1]$. This is a contradiction, because $\varnothing$ and $[0,1]$ are the only clopen subsets of $[0,1]$.
		%
	\end{proof}%
	\renewcommand{\qedsymbol}{$\blacksquare$}

	To proceed, we require an intuitive monotonicity claim. And to establish that claim, we need the following intuitive `responsiveness' claim.

	\begin{namedthm}[Responsiveness claim.]
		%
		\label{claim:mix:strict}
		%
		For any $\pi,\rho \in \Pi$ such that $\pi \succ \rho$ and any $\alpha \in (0,1)$, $\pi \succ \pi_\alpha \rho \succ \rho$.
		%
	\end{namedthm}

	\begin{proof}[Proof of the {\hyperref[claim:mix:solv]{responsiveness claim}}]
		%
		\renewcommand{\qedsymbol}{$\square$}
		Fix $\pi,\rho \in \Pi$ such that $\pi \succ \rho$ and an $\alpha \in (0,1)$; we will show that $\pi \succ \pi_\alpha \rho$, omitting the analogous argument for $\pi_\alpha \rho \succ \rho$.

		To that end, suppose toward a contradiction that $\pi_\alpha \rho \succsim \pi$. Then by the \hyperref[claim:mix:solv]{solvability claim} (recalling that $\pi \succ \rho$), there is a $\beta \in [0,1]$ such that $\pi \sim ( \pi_\alpha \rho )_\beta \rho = \pi_{\alpha\beta} \rho$. (The equality holds by property~\ref{mix:comp} in the definition of a mixture space.) In other words, the set
		%
		\begin{equation*}
			\mathcal{B} \coloneqq \left\{ \beta \in [0,1] :
			\pi \sim \pi_{\alpha\beta} \rho
			\right\}
		\end{equation*}
		%
		is non-empty. By mixture continuity, $\mathcal{B}$ is closed. Hence $\mathcal{B}$ has a least element, which we denote by $\beta_0$.

		It must be that $\beta_0>0$, since otherwise $\pi \sim \pi_0 \rho = \rho_1 \pi = \rho$, a contradiction with the fact that $\pi \succ \rho$. (The two equalities hold by properties~\ref{mix:sym} and \ref{mix:id} in the definition of a mixture space.)

		Since $\pi \sim \pi_{\alpha \beta_0} \rho$, independence implies that
		%
		\begin{equation*}
			\pi_\alpha \rho
			\sim ( \pi_{\alpha \beta_0} \rho )_\alpha \rho
			= \pi_{\alpha^2 \beta_0} \rho .
		\end{equation*}
		%
		Since $\pi_\alpha \rho \succsim \pi \succ \rho$ by hypothesis, it follows by the \hyperref[claim:mix:solv]{solvability claim} that there exists a $\gamma \in [0,1]$ such that
		%
		\begin{equation*}
			\pi
			\sim \bigl( \pi_{\alpha^2 \beta_0} \rho \bigr)_\gamma \rho
			= \pi_{\alpha^2 \beta_0 \gamma} \rho .
		\end{equation*}
		%
		Hence $\alpha \beta_0 \gamma$ belongs $\mathcal{B}$. But $\alpha \beta_0 \gamma < \beta_0$, and $\beta_0$ is by definition the least element of $\mathcal{B}$---a contradiction.
		%
	\end{proof}%
	\renewcommand{\qedsymbol}{$\blacksquare$}

	\begin{namedthm}[Monotonicity claim.]
		%
		\label{claim:mix:mon}
		%
		For any $\alpha,\beta \in [0,1]$, $\overline{\pi}_\alpha \underline{\pi} \succsim \overline{\pi}_\beta \underline{\pi}$ iff $\alpha \geq \beta$.
		%
	\end{namedthm}

	\begin{proof}[Proof of the {\hyperref[claim:mix:solv]{monotonicity claim}}]
		%
		\renewcommand{\qedsymbol}{$\square$}
		We must show that $\alpha=\beta$ implies $\overline{\pi}_\alpha \underline{\pi} \sim \overline{\pi}_\beta \underline{\pi}$ and that $\alpha > \beta$ implies $\overline{\pi}_\alpha \underline{\pi} \succ \overline{\pi}_\beta \underline{\pi}$. The former is immediate. For the latter, suppose that $\alpha>\beta$. If $\beta=0$, then by the \hyperref[claim:mix:strict]{responsiveness claim}, $\overline{\pi}_\alpha \underline{\pi} \succ \underline{\pi} = \overline{\pi}_\beta \underline{\pi}$. (Exactly why does the equality hold?) Suppose instead that $\beta>0$. By the \hyperref[claim:mix:strict]{responsiveness claim}, $\overline{\pi}_\alpha \underline{\pi} \succ \underline{\pi}$. Since $\beta/\alpha \in (0,1)$, applying the \hyperref[claim:mix:strict]{responsiveness claim} again yields
		%
		\begin{equation*}
			\overline{\pi}_\alpha \underline{\pi}
			\succ ( \overline{\pi}_\alpha \underline{\pi} )_{\beta/\alpha} \underline{\pi}
			= \overline{\pi}_\beta \underline{\pi} .
			\qedhere
		\end{equation*}
		%
	\end{proof}%
	\renewcommand{\qedsymbol}{$\blacksquare$}

	The \hyperref[claim:mix:solv]{solvability} and \hyperref[claim:mix:mon]{monotonicity} claims together imply that for every $\pi \in \Pi$, there exists exactly one $\alpha \in [0,1]$ such that $\pi \sim \overline{\pi}_\alpha \underline{\pi}$; we denote this $\alpha$ by $U(\pi)$. The function $U : \Pi \to [0,1]$ represents $\succsim$: for any $\pi,\rho \in \Pi$,
	%
	\begin{equation*}
		\pi \succsim \rho
		\quad \text{iff} \quad
		\overline{\pi}_{U(\pi)} \underline{\pi}
		\succsim \overline{\pi}_{U(\rho)} \underline{\pi}
		\quad \text{iff} \quad
		U(\pi) \geq U(\rho) ,
	\end{equation*}
	%
	where the first `iff' holds by definition of $U$, and the second `iff' holds by the \hyperref[claim:mix:mon]{monotonicity claim}.

	It remains only to show that $U$ is affine. To that end, fix any $\pi,\rho \in \Pi$ and $\alpha \in [0,1]$; we must show that $U(\pi_\alpha \rho) = \alpha U(\pi) + (1-\alpha) U(\rho)$. Observe that
	%
	\begin{equation*}
		\pi_\alpha \rho
		\sim ( \overline{\pi}_{U(\pi)} \underline{\pi} )_\alpha \rho
		\sim ( \overline{\pi}_{U(\pi)} \underline{\pi} )_\alpha ( \overline{\pi}_{U(\rho)} \underline{\pi} )
		= \overline{\pi}_{\alpha U(\pi) + (1-\alpha) U(\rho)} \underline{\pi} ,
	\end{equation*}
	%
	where the two `$\sim$'s hold by independence, and the equality holds by \Cref{proposition:mix_twodistrib} (\cpageref{proposition:mix_twodistrib}). Hence $U(\pi_\alpha \rho) = \alpha U(\pi) + (1-\alpha) U(\rho)$ by definition of $U$.

	Finally, to prove the second (`furthermore') claim in the \hyperref[theorem:mix]{mixture-space theorem}, suppose that $U,V : \Pi \to \R$ are both affine and both represent $\succsim$, and define
	%
	\begin{equation*}
		a \coloneqq \frac{ U\left(\overline{\pi}\right) - U\left(\underline{\pi}\right) }{ V\left(\overline{\pi}\right) - V\left(\underline{\pi}\right) }
		> 0
		\quad \text{and} \quad
		b \coloneqq U\left(\underline{\pi}\right) - a V\left(\underline{\pi}\right) ;
	\end{equation*}
	%
	we claim that $U = a V + b$. To that end, fix an arbitrary $\pi \in \Pi$, and let $\alpha$ be the unique $\beta \in [0,1]$ such that $\pi \sim \overline{\pi}_\beta \underline{\pi}$. (Why does $\alpha$ exist? Why is it unique?) Then
	%
	\begin{align*}
		U(\pi)
		&= U\left( \overline{\pi}_\alpha \underline{\pi} \right)
		\\
		&= \alpha U\left(\overline{\pi}\right)
		+ (1-\alpha) U\left(\underline{\pi}\right)
		\\
		&= \alpha \left[ a V\left(\overline{\pi}\right) + b \right]
		+ (1-\alpha) \left[ a V\left(\underline{\pi}\right) + b \right]
		\\
		&= a \left[ \alpha V\left(\overline{\pi}\right)
		+ (1-\alpha) V\left(\underline{\pi}\right) \right] + b
		\\
		&= a V\left( \overline{\pi}_\alpha \underline{\pi} \right) + b
		\\
		&= a V(\pi) + b ,
	\end{align*}
	%
	where the first (final) equality holds since $U$ ($V$) represents $\succsim$, the second (penultimate) equality holds since $U$ ($V$) is affine, and the third equality holds since $U\left(\overline{\pi}\right) = a V\left(\overline{\pi}\right) + b$ and $U\left(\underline{\pi}\right) = a V\left(\underline{\pi}\right) + b$ by construction of $a$ and $b$.
	%
\end{proof}

The sketch proof above relies on the extra assumption that there are best and worst elements of $\Pi$, $\overline{\pi}$ and $\underline{\pi}$. Without this assumption, we would instead fix an arbitrary pair $\overline{\pi},\underline{\pi} \in \Pi$ such that $\overline{\pi} \succ \underline{\pi}$, and assign utility values $U(\pi)$ as above to all elements $\pi \in \Pi$ such that $\overline{\pi} \succsim \pi \succsim \underline{\pi}$. The only change is that we must now somehow extend this formula to those $\pi \in \Pi$ that are better than $\overline{\pi}$ or worse than $\underline{\pi}$. Doing this is actually pretty straightforward.




%%%%%%%%%%%%%%%%%%%%%%
%%%%%%%%%%%%%%%%%%%%%%
%%%%%%%%%%%%%%%%%%%%%%
\chapter{Risk}
\label{ch_risk}
%%%%%%%%%%%%%%%%%%%%%%
%%%%%%%%%%%%%%%%%%%%%%
%%%%%%%%%%%%%%%%%%%%%%

% Copyright (c) 2025 Carl Martin Ludvig Sinander.

% This program is free software: you can redistribute it and/or modify
% it under the terms of the GNU General Public License as published by
% the Free Software Foundation, either version 3 of the License, or
% (at your option) any later version.

% This program is distributed in the hope that it will be useful,
% but WITHOUT ANY WARRANTY; without even the implied warranty of
% MERCHANTABILITY or FITNESS FOR A PARTICULAR PURPOSE. See the
% GNU General Public License for more details.

% You should have received a copy of the GNU General Public License
% along with this program. If not, see <https://www.gnu.org/licenses/>.

%%%%%%%%%%%%%%%%%%%%%%%%%%%%%%%%%%%%%%%%%%%%%%%%%%%%%%%%%%%%%%%%%%%%%%%

In this chapter, we study quantifiable or `objective' uncertainty: \emph{risk.} To be more precise, we study the case in which each uncertain prospect is associated with a lottery, meaning a probability distribution over alternatives, and these probabilities are known both to the decision-maker and to us (the economic modellers). Under these assumptions (and given that the decision-maker cares only about which alternative arises, by definition `alternatives'), we can identify each prospect with its associated lottery, reducing choice among uncertain prospects to choice among (or preferences over) lotteries.



%%%%%%%%%%%%%%%%%%%%%%%%%%%%%%%%%%%
%%%%%%%%%%%%%%%%%%%%%%%%%%%%%%%%%%%
\section{Preferences over lotteries}
\label{risk:lotteries}
%%%%%%%%%%%%%%%%%%%%%%%%%%%%%%%%%%%
%%%%%%%%%%%%%%%%%%%%%%%%%%%%%%%%%%%

\emph{This section is drawn pretty much verbatim from one of my papers \parencite{oo}.}

There is a non-empty set $X$ of alternatives, with generic elements $x,y,z,w \in X$. We consider simple lotteries, meaning (probability mass functions of) finitely supported probability distributions over $X$. Formally, a \emph{simple lottery} is a function $p : X \to [0,1]$ such that $\supp(p) \coloneqq \{ x \in X : p(x) > 0 \}$ is finite and $\sum_{x \in \supp(p)} p(x) = 1$. We write $\Delta^0(X)$ for the set of all simple lotteries, with generic elements $p,q,r \in \Delta^0(X)$. By a standard abuse, the lottery in $\Delta^0(X)$ that is degenerate at $x \in X$ is denoted simply `$x$'.

A \emph{preference} is a complete and transitive binary relation on $\Delta^0(X)$. A decision-maker's preference is, at least in principle, an empirical object: it can be recovered from (sufficiently rich) choice data.

Comparative risk-aversion is defined as follows.

\begin{definition}[\cite{Yaari1969}]
	%
	\label{definition:lra}
	%
	For any two preferences $\succsim$ and $\succsim'$, $\succsim$ is called \emph{less risk-averse than} $\succsim'$ if and only if for each alternative $x \in X$ and each simple lottery $p \in \Delta^0(X)$, $x \succsim \mathrel{(\succ)} p$ implies $x \succsim' \mathrel{(\succ')} p$.
	%
\end{definition}

`Expected-utility' preferences are those which can be viewed as arising from maximisation of the expectation (under the lottery at hand) of some function $u : X \to \R$.

\begin{definition}[\cite{Bernoulli1738}]
	%
	\label{definition:eu}
	%
	A preference $\succsim$ is called \emph{expected-utility} if and only if there exists a function $u : X \to \R$ such that for any simple lotteries $p,q \in \Delta^0(X)$, $p \succsim q$ if and only if $\int u \dd p \geq \int u \dd q$. (Here `$\int u \dd p$' is shorthand for $\sum_{x \in \supp(p)} p(x) u(x)$.)
	%
\end{definition}

The function $u : X \to \R$ is called a \emph{risk attitude} (or `vNM utility function', or `Bernoulli utility function'), and is said to \emph{represent} $\succsim$.

Non-expected-utility preferences arise naturally in many contexts. Psychological reasons for this are often emphasised, e.g. the \textcite{Allais1953} thought experiment. But squarely economic forces can also easily produce non-expected-utility behaviours. There are many examples of this; perhaps the most economically fundamental is the following.

\begin{exercise}
	%
	\label{exercise:eu_choice}
	%
	Consider a decision-maker who must not only choose a lottery, but must also choose an action. Imagine, for example, a manager who chooses among projects (risky prospects, modelled as lotteries) and, after choosing her project, chooses how to operate the project, e.g. what staff to employ on her team and how to organise them. The operational options are modelled as a non-empty set $A$ of actions. Suppose that for each given action $a \in A$, the decision-maker has expected-utility preferences: she evaluates each simple lottery $p \in \Delta^0(X)$ at $\int u_a \dd p$, for some risk attitude $u_a : X \to \R$. Then taking into account optimal action choice, she evaluates each lottery $p \in \Delta^0(X)$ at $U^{(A,(u_a)_{a \in A})}(p) \coloneqq \max_{a \in A} \int u_a \dd p$. (That is: her preference $\succsim$ is such that for any simple lotteries $p,q \in \Delta^0(X)$, $p \succsim q$ holds if and only if $U^{(A,(u_a)_{a \in A})}(p) \geq U^{(A,(u_a)_{a \in A})}(q)$.)

	\begin{enumerate}[label=(\alph*)]
	
		\item Remind yourself of \hyperref[exercise:mix_lottsimple_affine]{\Cref*{exercise:mix_lottsimple}} (\cref{ch_mix}, \cpageref{exercise:mix_lottsimple_affine}), which a preference is expected-utility if and only if it admits a representation $U : \Delta^0(X) \to \R$ that is affine.

		\item Show that $U^{(A,(u_a)_{a \in A})}$ is convex.%
			\footnote{\label{footnote:convex_max}There is a converse: any continuous convex function $\Delta^0(X) \to \R$ may be approximated arbitrarily well by $U^{(A,(u_a)_{a \in A})}$ for some non-empty set $A$ and family $(u_a)_{a \in A}$ of functions $X \to \R$. See the proof of \hyperref[theorem:blackwell]{Blackwell's theorem} in \cref{info:blackwell} below.}

		\item Under what conditions is $U^{(A,(u_a)_{a \in A})}$ affine?
	
	\end{enumerate}
	%
\end{exercise}



%%%%%%%%%%%%%%%%%%%%%%%%%%%%%%%%%%%
%%%%%%%%%%%%%%%%%%%%%%%%%%%%%%%%%%%
\section{The von Neumann--Morgenstern theorem}
\label{risk:vnm}
%%%%%%%%%%%%%%%%%%%%%%%%%%%%%%%%%%%
%%%%%%%%%%%%%%%%%%%%%%%%%%%%%%%%%%%

For any simple lotteries $p,q \in \Delta^0(X)$ and a constant $\alpha \in [0,1]$, we write $\alpha p + (1-\alpha) q$ for the simple lottery defined by
%
\begin{equation*}
	(\alpha p + (1-\alpha) q)(x) \coloneqq \alpha p(x) + (1-\alpha) q(x)
	\quad \text{for each $x \in X$.}
\end{equation*}
%
This can, but needn't, be interpreted as the compound lottery obtained by first flipping an $\alpha$-biased coin, then drawing an alternative from $p$ in case of heads and from $q$ in case of tails.

\begin{definition}
	%
	\label{definition:vnm_indep}
	%
	Given a non-empty set $X$, a preference $\succsim$ on $\Delta^0(X)$ satisfies \emph{independence} iff for all $p,r,q \in \Delta^0(X)$ and $\alpha \in [0,1]$, $p \sim q$ implies $\alpha p + (1-\alpha) r \sim \alpha q + (1-\alpha) r$.
	%
\end{definition}

\begin{definition}
	%
	\label{definition:vnm_continuity}
	%
	Given a non-empty set $X$, a preference $\succsim$ on $\Delta^0(X)$ satisfies \emph{mixture continuity} iff for all $p,q,r \in \Delta^0(X)$ such that $p \succsim q \succsim r$, the sets $\{ \alpha \in [0,1] : \alpha p + (1-\alpha) r \succsim q \}$ and $\{ \alpha \in [0,1] : \alpha p + (1-\alpha) r \precsim q \}$ are closed in $[0,1]$.
	%
\end{definition}

These are exactly the independence and mixture continuity concepts from the general mixture-space context of \cref{ch_mix}, specialised to the particular mixture space $\Delta^0(X)$ equipped with the operation $(p,\alpha,q) \mapsto \alpha p + (1-\alpha) q$.

The following result characterises the behavioural content and identification properties of the expected-utility model.

\begin{namedthm}[von Neumann--Morgenstern theorem {\normalfont\parencite{VonneumannMorgenstern1947}}.]
	%
	\label{theorem:vNM}
	%
	A preference is expected-utility if and only if it satisfies independence and mixture continuity. Furthermore, if two risk attitudes $u,v : X \to \R$ represent the same expected-utility preference, then there exist $a>0$ and $b \in \R$ such that $u = a v + b$.
	%
\end{namedthm}

\begin{exercise}
	%
	\label{exercise:vNM_proof}
	%
	Prove it! (Use \cref{ch_mix}.)
	%
\end{exercise}

As in the \hyperref[theorem:mix]{mixture-space theorem} (\cpageref{theorem:mix}, \cref{ch_mix}), there are many variations on the theorem in which either independence or mixture continuity is replaced by another (qualitatively similar) property.

\begin{exercise}
	%
	\label{exercise:vNM_indp}
	%
	Give examples of the following.

	\begin{enumerate}[label=(\alph*)]
	
		\item A preference that satisfies mixture continuity but not independence.

		\item A preference that satisfies independence but not mixture continuity.
	
	\end{enumerate}
	%
\end{exercise}

\begin{namedthm}[\Cref*{exercise:eu_choice} {\normalfont (continued)}.]
	%
	\label{exercise:eu_choice_axioms}
	%
	Read \textcite{CerreiavioglioDillenbergerOrtoleva2015}. Characterise the behavioural content of the expected-utility-with-choice model: that is, identify a set of properties such that a preference $\succsim$ satisfies these properties if and only if there exists a non-empty set $A$ and a collection $(u_a)_{a \in A}$ of maps $X \to \R$ such that $U^{(A,(u_a)_{a \in A})}$ represents $\succsim$ (that is, for any simple lotteries $p,q \in \Delta^0(X)$, $p \succsim q$ holds if and only if $U^{(A,(u_a)_{a \in A})}(p) \geq U^{(A,(u_a)_{a \in A})}(q)$).
	%
\end{namedthm}



%%%%%%%%%%%%%%%%%%%%%%%%%%%%%%%%%%%
%%%%%%%%%%%%%%%%%%%%%%%%%%%%%%%%%%%
\section{Pratt's theorem}
\label{risk:pratt}
%%%%%%%%%%%%%%%%%%%%%%%%%%%%%%%%%%%
%%%%%%%%%%%%%%%%%%%%%%%%%%%%%%%%%%%

\emph{This section is drawn pretty much verbatim from one of my papers \parencite{oo}.}

The following theorem characterises `less risk-averse than' (defined on \cpageref{definition:lra} above) for expected-utility preferences:

\begin{namedthm}[Pratt's theorem, part~1 {\normalfont\parencite{Pratt1964}}.]
	%
	\label{theorem:pratt}
	%
	For a non-empty set $X$ and functions $u,v : X \to \R$, the following are equivalent:

	\begin{enumerate}[label=(\Alph*)]
	
		\item \label{item:pratt:lra} $u$ is less risk-averse than $v$, i.e. for any alternative $x \in X$ and simple lottery $p \in \Delta^0(X)$, $u(x) \geq \mathrel{(>)} \int u \dd p$ implies $v(x) \geq \mathrel{(>)} \int v \dd p$.

		\item \label{item:pratt:trans} There exists an increasing convex function $\phi : \co(v(X)) \to \R$ that is strictly increasing on $v(X)$ and satisfies $u = \phi \circ v$.

		\item \label{item:pratt:curv} The following two properties hold:
		%
		\begin{enumerate}[label=(\Roman*),topsep=0em]
		
			\item \label{item:pratt:curv:ordequiv} For any $x,y \in X$, $u(x) \geq \mathrel{(>)} u(y)$ implies $v(x) \geq \mathrel{(>)} v(y)$.

			\item \label{item:pratt:curv:compress} For any alternatives $x,y,z \in X$, if $u(x) < u(y) < u(z)$, then
			%
			\begin{equation*}
				\frac{u(z)-u(y)}{u(y)-u(x)}
				\geq \frac{v(z)-v(y)}{v(y)-v(x)} .
			\end{equation*}
		
		\end{enumerate}
	
	\end{enumerate}
	%
\end{namedthm}

Property~\ref{item:pratt:trans} asserts strict monotonicity of $\phi$ only on $v(X)$. It need not be possible to choose $\phi$ to be strictly increasing on its full domain $\co(v(X))$, nor need it be possible to choose $\phi$ to be continuous. To see why, consider the following two examples.

\begin{example}
	%
	\label{example:pratt_disc}
	%
	Consider $X \coloneqq [0,1]$ and $u,v : X \to \R$, where $v$ is the identity, $u=v$ on $[0,1)$, and $u(1)=2$. Then $u$ is less risk-averse than $v$, but the only $\phi : \co(v(X)) \to \R$ that satisfies $u = \phi \circ v$ is discontinuous at $\max v(X) = 1$.
	%
\end{example}

\begin{example}
	%
	\label{example:pratt_str}
	%
	Consider $X \coloneqq [1,2]$ and $u,v : X \to \R$, where $u$ is the identity, $v(1)=0$, and $v=u$ on $(1,2]$. Then $u$ is less risk-averse than $v$, but the only increasing $\phi : \co(v(X)) \to \R$ that satisfies $u = \phi \circ v$ is constant on $[0,1]$.
	%
\end{example}

\Cref{example:pratt_str} shows that the strict monotonicity of $\phi$ on $v(X)$ in property~\ref{item:pratt:trans} in \hyperref[theorem:pratt]{Pratt's theorem} cannot be strengthened to strict monotonicity on $\co(v(X))$. It can, however, be strengthened to strict monotonicity on a large subset of $\co(v(X))$, as the next result shows.

\begin{definition}
	%
	\label{definition:cotwo}
	%
	For any set $A \subseteq \R$, define $\cotwo(A) \subseteq A$ and $\inftwo A \in \cl(A)$ by
	%
	\begin{align*}
		&\begin{aligned}
			\cotwo(A) &\coloneqq \co( A \setminus \{\inf A\} ) \cup \{\inf A\}
			\\
			\inftwo A &\coloneqq \inf ( A \setminus \{\inf A\} )
		\end{aligned}
		&&\Biggr\} \quad \text{if $\inf A < \inf( A \setminus \{\inf A\} ) \notin A$}
		\\
		&\begin{aligned}
			\cotwo(A) &\coloneqq \co(A)
			\\
			\inftwo A &\coloneqq \inf A
		\end{aligned}
		&&\Biggr\} \quad \text{otherwise.}
	\end{align*}
	%
\end{definition}

Evidently $A \subseteq \cotwo(A) \subseteq \co(A)$ and $\co(A) \setminus \cotwo(A) = (\inf A,\inftwo A]$. If $A$ is convex, then $A = \cotwo(A) = \co(A)$.

\begin{lemma}[\cite{oo}]
	%
	\label{lemma:greatest_phi}
	%
	Fix a non-empty set $X$ and functions $u,v : X \to \R$, and let $\Phi$ be the set of all increasing convex functions $\phi : \co(v(X)) \to \R$ that are strictly increasing on $v(X)$ and satisfy $u = \phi \circ v$. If $\Phi$ is not empty, then it has a pointwise greatest element, which is strictly increasing on $\cotwo(v(X))$ and affine on each maximal interval of $\co(v(X)) \setminus v(X)$.
	%
\end{lemma}

\begin{proof}
	%
	Assume that $\Phi$ is non-empty. Note that there is exactly one function $\phi_0 : v(X) \to \R$ such that $u = \phi_0 \circ v$, and that this $\phi_0$ is strictly increasing. Define $\phi : \co(v(X)) \to \R$ by $\phi(k) \coloneqq \sup_{\psi \in \Phi} \psi(k)$ for each $k \in \co(v(X))$. By inspection, $\phi$ is increasing and convex, and is strictly increasing on $v(X)$ since $\phi=\phi_0$ on $v(X)$; thus $\phi \in \Phi$. Obviously $\phi \geq \psi$ for every $\psi \in \Phi$. Since each $\psi \in \Phi$ is convex and satisfies $\psi = \phi_0$ on $v(X)$, $\phi$ is affine on each maximal interval of $\co(v(X)) \setminus v(X)$.

	To show that $\phi$ is strictly increasing on $\cotwo(v(X))$, fix any $k'<\ell'$ in $\cotwo(v(X))$; we must show that $\phi(k') < \phi(\ell')$. It suffices to find $k<\ell$ in $\co(v(X))$ such that $k \leq k'$, $\ell \leq \ell'$, and $\phi(k)<\phi(\ell)$, since then
	%
	\begin{equation*}
		\phi(\ell') - \phi(k')
		\geq (\ell'-k') \frac{\phi(\ell) - \phi(k)}{\ell-k}
		> 0 ,
	\end{equation*}
	%
	where the weak inequality holds since $\phi$ is convex. Write $m_1 \coloneqq \inf v(X)$ and $m_2 \coloneqq \inf( v(X) \setminus \{m_1\} )$, and note that $k' \geq m_1 < \ell' \geq m_2$.

	We consider four cases. In the first three, we find $k<\ell$ in $v(X)$ such that $k \leq k'$ and $\ell \leq \ell'$; then $\phi(k)<\phi(\ell)$ since $\phi$ is strictly increasing on $v(X)$. In the final case, we directly choose $k<\ell$ in $\co(v(X))$ to satisfy $\phi(k)<\phi(\ell)$.

	\smallskip

	\noindent
	\emph{Case~1: $m_1 \notin v(X)$.} Here $k' > m_1$ and $\ell' > m_1 = m_2$, so choosing $k<\ell$ in $v(X)$ sufficiently close to $m_1$ ensures that $k \leq k'$ and $\ell \leq \ell'$.

	\smallskip

	\noindent
	\emph{Case~2: $v(X) \ni m_1 = m_2$.} 
	Here $m_1 \in \cl( v(X) \setminus \{m_1\} )$, so choosing $k \coloneqq m_1$ and $\ell \in v(X) \setminus \{m_1\}$ sufficiently close to $m_1$ ensures that $k \leq k'$ and $\ell \leq \ell'$.

	\smallskip

	\noindent
	\emph{Case~3: $m_1 < m_2 \notin v(X)$.}
	Here $\varnothing \neq \co(v(X)) \setminus \cotwo(v(X)) = (m_1,m_2]$, whence $m_1 \in v(X)$ and $\ell'>m_2$, so choosing $k \coloneqq m_1$ and $\ell \in v(X) \setminus [m_1,m_2]$ sufficiently close to $m_2$ ensures that $k \leq k'$ and $\ell \leq \ell'$.

	\smallskip

	\noindent
	\emph{Case~4: $m_1 < m_2 \in v(X)$.}
	Here $m_1 \in v(X)$, so $\phi(m_1)<\phi(m_2)$, which since $\phi$ is affine on $[m_1,m_2]$ implies that $\phi$ is strictly increasing on $[m_1,m_2]$, so that $k \coloneqq m_1$ and $\ell \coloneqq \min\{\ell',m_2\}$ satisfy $\phi(k)<\phi(\ell)$.
	%
\end{proof}


\begin{proof}[Proof of {\hyperref[theorem:pratt]{Pratt's theorem (part~1)}}]
	%
	We shall prove that \ref{item:pratt:trans} implies \ref{item:pratt:lra} implies \ref{item:pratt:curv} implies \ref{item:pratt:trans}.

	To prove that \ref{item:pratt:trans} implies \ref{item:pratt:lra}, suppose there exists an increasing convex function $\psi : \co(v(X)) \to \R$ that is strictly increasing on $v(X)$ and satisfies $u = \psi \circ v$. Then by \Cref{lemma:greatest_phi}, there exists an increasing convex function $\phi : \co(v(X)) \to \R$ that is strictly increasing on $\cotwo(v(X))$ and satisfies $u = \phi \circ v$. Fix an alternative $x \in X$ and a simple lottery $p \in \Delta^0(X)$, and suppose that $\int v \dd p \geq \mathrel{(>)} v(x)$; we must show that $\int u \dd p \geq \mathrel{(>)} u(x)$. If $\int v \dd p \in \cotwo(v(X))$, then
	%
	\begin{equation*}
		\int u \dd p
		= \int (\phi \circ v) \dd p
		\geq \phi\left( \int v \dd p \right)
		\geq \mathrel{(>)} \phi(v(x))
		= u(x) ,
	\end{equation*}
	%
	where the first inequality holds (by Jensen's inequality) since $\phi$ is convex, and the second holds since $\phi$ is strictly increasing on $\cotwo(v(X)) \supseteq v(X) \ni v(x)$. If instead $\int v \dd p \notin \cotwo(v(X))$, then
	%
	\begin{equation*}
		\int v \dd p
		\in \co(v(X)) \setminus \cotwo(v(X))
		= \bigl( \inf v(X), \inf\bigl( v(X) \setminus \{\inf v(X)\} \bigr) \bigr] ,
	\end{equation*}
	%
	so writing $Y \coloneqq \{ y \in X : v(y) = \inf v(X) \}$, we see that $\int v \dd p \geq \mathrel{(>)} v(x)$ implies $x \in Y$ (and $p(X \setminus Y)>0$), whence
	%
	\begin{align*}
		\int u \dd p
		={}& p(Y) \phi(v(x)) 
		+ \int_{X \setminus Y} (\phi \circ v) \dd p
		\\
		\geq \mathrel{(>)}{}& p(Y) \phi(v(x))
		+ p(X \setminus Y) \phi(v(x))
		= u(x) 
	\end{align*}
	%
	since $\phi$ is strictly increasing on $v(X)$.

	To prove that \ref{item:pratt:lra} implies \ref{item:pratt:curv}, suppose that $u$ is less risk-averse than $v$. It follows immediately (by considering degenerate lotteries $p \in \Delta^0(X)$) that property~\ref{item:pratt:curv}\ref{item:pratt:curv:ordequiv} holds.
	To show that property~\ref{item:pratt:curv}\ref{item:pratt:curv:compress} holds, suppose toward a contradiction that it does not: there are $x,y,z \in X$ such that $u(x) < u(y) < u(z)$ and
	%
	\begin{equation*}
		\frac{u(z)-u(y)}{u(y)-u(x)}
		< \frac{v(z)-v(y)}{v(y)-v(x)} .
	\end{equation*}
	%
	By replacing $u$ with $a u + b$ for some $a > 0$ and $b \in \R$ if necessary, we may assume without loss of generality that $u(x) = v(x)$ and $u(y) = v(y)$, so that $u(z) < v(z)$.
	Define a simple lottery $p \in \Delta^0(X)$ by $p(x) \coloneqq [ u(z) - u(y) ] / [ u(z) - u(x) ]$, $p(z) \coloneqq 1-p(x)$, and $p(w) \coloneqq 0$ for every $w \in X \setminus \{x,z\}$.
	Then $u(y) = \int u \dd p$ and
	%
	\begin{equation*}
		v(y)
		= u(y)
		= p(x) u(x) + p(z) u(z)
		< p(x) v(x) + p(z) v(z)
		= \int v \dd p ,
	\end{equation*}
	%
	a contradiction with the fact that $u$ is less risk-averse than $v$.

	To prove that \ref{item:pratt:curv} implies \ref{item:pratt:trans}, suppose that $u$ satisfies properties~\ref{item:pratt:curv}\ref{item:pratt:curv:ordequiv} and \ref{item:pratt:curv}\ref{item:pratt:curv:compress}; we must identify an increasing convex function $\phi : \co(v(X)) \to \R$ that is strictly increasing on $v(X)$ and satisfies $u = \phi \circ v$. By property~\ref{item:pratt:curv}\ref{item:pratt:curv:ordequiv}, there exists a strictly increasing $\psi : v(X) \to \R$ such that $u = \psi \circ v$.
	Define $\overline{\psi} : \cl(v(X)) \cap \co(v(X)) \to \R$ by
	%
	\begin{equation*}
		\overline{\psi}(k)
		\coloneqq
		\begin{cases}
			\psi(k) & \text{if $k \in v(X)$} \\
			\lim_{\ell \to k} \psi(\ell) & \text{if $k \in [ \cl(v(X)) \cap \co(v(X)) ] \setminus v(X)$,} 
		\end{cases}
	\end{equation*}
	%
	where the limit exists (in $\R$) by the monotonicity of $\psi$ and property~\ref{item:pratt:curv}\ref{item:pratt:curv:compress}.%
		\footnote{Fix any $k \in [ \cl(v(X)) \cap \co(v(X)) ] \setminus v(X)$. Since $k \in \cl(v(X))$, there is a monotone sequence in $v(X)$ that converges to $k$, which since $\psi$ is increasing implies that either the left-hand limit $\psi(k-)$ or the right-hand limit $\psi(k+)$ must exist in $\R \cup \{-\infty,+\infty\}$. Since $k \in \co(v(X))$, there are $m_0,m_1 \in v(X)$ such that $m_0 \leq k \leq m_1$. Since $\psi$ is increasing, it is bounded on $[m_0,m_1] \intersect v(X)$. Hence $\psi(k-)$ is finite if it exists, and likewise for $\psi(k+)$.

		It remains only to show that if $\psi(k-)$ and $\psi(k+)$ both exist, then they are equal. We have $\psi(k-) \leq \psi(k+)$ since $\psi$ is increasing. To show that $\psi(k-) \geq \psi(k+)$, suppose toward a contradiction that $\psi(k-) < \psi(k+)$, and fix an $m \in v(X)$ such that $k<m$. Then we can choose $\ell,k' \in v(X)$ arbitrarily close to $k$ and satisfying $\ell<k<k'<m$, and by doing so we may make $[ \psi(k') - \psi(\ell) ] / [k'-\ell]$ arbitrarily large. Since $\psi$ is increasing, it is bounded on a neighbourhood of $k$, so $\ell,k' \in v(X)$ can be chosen so that $[ \psi(m) - \psi(k') ] / [m-k']$ is bounded. Hence $\ell,k' \in v(X)$ can be chosen so that $[ \psi(k') - \psi(\ell) ] / [k'-\ell] > [ \psi(m) - \psi(k') ] / [m-k']$, a contradiction with property~\ref{item:pratt:curv}\ref{item:pratt:curv:compress}.}
	Let $\phi$ be the (unique) function $\co(v(X)) \to \R$ that matches $\overline{\psi}$ on $\cl(v(X)) \cap \co(v(X))$ and is affine on the closure of each maximal interval of $\co(v(X)) \setminus v(X)$.
	Evidently $\phi$ is increasing, and $\phi$ is convex since by property~\ref{item:pratt:curv}\ref{item:pratt:curv:compress},
	%
	\begin{equation*}
		\frac{\phi(\ell)-\phi(k)}{\ell-k}
		\leq \frac{\phi(m)-\phi(\ell)}{m-\ell}
		\quad \text{for all $k < \ell < m$ in $\co(v(X))$.}
	\end{equation*}
	%
	Since $\phi=\psi$ on $v(X)$, $\phi$ is strictly increasing on $v(X)$, and $u = \phi \circ v$.
	%
\end{proof}

\begin{exercise}[\cite{oo}]
	%
	\label{exercise:oo}
	%
	Let $X$ be a non-empty finite set of alternatives. Consider a decision-maker who has expected-utility preferences over lotteries $\Delta(X)$, with risk attitude $v : X \to \R$. Suppose that she additionally has access to a possibly uncertain outside option, with full-support distribution $\overline{p} \in \Delta(X)$. The decision-maker decides ex post whether to exercise her outside option: that is, after choosing (ex ante) a lottery $p \in \Delta(X)$, an alternative (`the inside option') is drawn from $p$, an alternative is independently drawn from $\overline{p}$ (`the outside option'), and the decision-maker takes home whichever of the two she prefers. Write $\succsim$ for the decision-maker's preference (a complete and transitive binary relation on $\Delta(X)$).

	\begin{enumerate}[label=(\alph*)]

		\item Write down a utility representation of $\succsim$.

		\item Show that $\succsim$ is expected-utility. Write down an explicit expression for its risk attitude $u$ in terms of the model's primitives (namely, $v$ and $\overline{p}$).

		\item Let $\succsim^\star$ be the expected-utility preference with risk attitude $v$ (that is, for any lotteries $p,q \in \Delta(X)$, $p \succsim^\star q$ if and only if $\int v \dd p \geq \int v \dd q$). We can think of $\succsim^\star$ as the decision-maker's `true' risk attitude, before economic influences in the environment (in particular, the presence of the outside option) are taken into account. Prove that (whatever the true risk attitude $v$ and outside-option distribution $\overline{p}$,) $\succsim$ is less risk-averse than $\succsim^\star$.

	\end{enumerate}
	%
\end{exercise}





%%%%%%%%%%%%%%%%%%%%%%
%%%%%%%%%%%%%%%%%%%%%%
%%%%%%%%%%%%%%%%%%%%%%
\chapter{Monetary risk}
\label{ch_mone}
%%%%%%%%%%%%%%%%%%%%%%
%%%%%%%%%%%%%%%%%%%%%%
%%%%%%%%%%%%%%%%%%%%%%

% Copyright (c) 2025 Carl Martin Ludvig Sinander.

% This program is free software: you can redistribute it and/or modify
% it under the terms of the GNU General Public License as published by
% the Free Software Foundation, either version 3 of the License, or
% (at your option) any later version.

% This program is distributed in the hope that it will be useful,
% but WITHOUT ANY WARRANTY; without even the implied warranty of
% MERCHANTABILITY or FITNESS FOR A PARTICULAR PURPOSE. See the
% GNU General Public License for more details.

% You should have received a copy of the GNU General Public License
% along with this program. If not, see <https://www.gnu.org/licenses/>.

%%%%%%%%%%%%%%%%%%%%%%%%%%%%%%%%%%%%%%%%%%%%%%%%%%%%%%%%%%%%%%%%%%%%%%%


In this chapter, we continue our study of risk, specialising to the economically important case in which the alternatives are monetary. (Mathematically, what matters is that they are real numbers.)



%%%%%%%%%%%%%%%%%%%%%%%%%%%%%%%%%%%
%%%%%%%%%%%%%%%%%%%%%%%%%%%%%%%%%%%
\section{The monetary Pratt theorem}
\label{mone:pratt_money}
%%%%%%%%%%%%%%%%%%%%%%%%%%%%%%%%%%%
%%%%%%%%%%%%%%%%%%%%%%%%%%%%%%%%%%%

\emph{This section is drawn pretty much verbatim from one of my papers \parencite{oo}.}

If the alternatives are monetary prizes (i.e. $X \subseteq \R$ with strictly increasing utility $u : X \to \R$), then given some smoothness, `less risk-averse than' is characterised by a differential inequality:

\begin{namedthm}[Pratt's theorem, part~2.]
	%
	\label{theorem:pratt_diff}
	%
	For a non-empty open convex subset $X$ of $\R$ and twice continuously differentiable functions $u,v : X \to \R$ satisfying $u' > 0 < v'$, $u$ is less risk-averse than $v$ if and only if $u''/u' \geq v''/v'$.
	%
\end{namedthm}

The ratio $u''/u'$ is a measure of `how convex' the function $u$ is. More precisely, it is a measure of \emph{local} curvature: $u''(x)/u'(x)$ quantifies `how convex' $u$ is near $x \in X$. This measure has the advantage that it is invariant under positive affine transformations of $u$: if $u = av+b$ for some $a>0$ and $b \in \R$, then $u''/u' = v''/v'$. Economists often work with $-u''/u'$ rather than $u''/u'$, and call it the `Arrow--Pratt index (or coefficient)' of (absolute) risk-aversion, after \textcite{Arrow1965,Pratt1964}.

\begin{proof}
	%
	By \hyperref[theorem:pratt]{part~1 of Pratt's theorem} (\cpageref{theorem:pratt}), it suffices to show that property~\ref{item:pratt:curv} holds if and only if $u''/u' \geq v''/v'$. Note that property~\ref{item:pratt:curv}\ref{item:pratt:curv:ordequiv} holds since $u$ and $v$ are strictly increasing (as $u' > 0 < v'$). Hence by \hyperref[theorem:pratt]{Pratt's theorem (part~1}, \cpageref{theorem:pratt}), what must be shown is that property~\ref{item:pratt:curv}\ref{item:pratt:curv:compress} holds if and only if $u''/u' \geq v''/v'$.

	Suppose that property~\ref{item:pratt:curv}\ref{item:pratt:curv:compress} holds. Then for any $x<y<z<w$ in $X$,
	%
	\begin{align*}
		\frac{u(w)-u(z)}{u(y)-u(x)}
		&= \frac{u(w)-u(z)}{u(z)-u(y)}
		\times \frac{u(z)-u(y)}{u(y)-u(x)}
		\\
		&\geq \frac{v(w)-v(z)}{v(z)-v(y)}
		\times \frac{v(z)-v(y)}{v(y)-v(x)}
		= \frac{v(w)-v(z)}{v(y)-v(x)} .
	\end{align*}
	%
	Hence for each $x \in X$,
	%
	\begin{align*}
		\frac{u''(x)}{u'(x)}
		&= \left. \frac{\dd}{\dd y} \ln( u'(y) ) \right|_{y=x}
		= \lim_{\eps \searrow 0} \frac{1}{\eps}
		\ln\left( \frac{u'(x+\eps)}{u'(x)} \right)
		\\
		&= \lim_{\eps \searrow 0}
		\frac{1}{\eps} \ln\left( \frac
		{ \lim_{\delta \searrow 0} \frac{1}{\delta}
		\left[ u(x+\eps+\delta) - u(x+\eps) \right] }
		{ \lim_{\delta \searrow 0} \frac{1}{\delta}
		\left[ u(x+\delta) - u(x) \right] }
		\right)
		\\
		&= \lim_{\eps \searrow 0}
		\lim_{\delta \searrow 0} 
		\frac{1}{\eps} \ln\left( \frac
		{ u(x+\eps+\delta) - u(x+\eps) }
		{ u(x+\delta) - u(x) }
		\right) 
		\\
		&\geq \lim_{\eps \searrow 0}
		\lim_{\delta \searrow 0} 
		\frac{1}{\eps} \ln\left( \frac
		{ v(x+\eps+\delta) - v(x+\eps) }
		{ v(x+\delta) - v(x) }
		\right)
		= \frac{v''(x)}{v'(x)} .
	\end{align*}

	Conversely, suppose that $u''/u' \geq v''/v'$. Since $u' > 0 < v'$, we have
	%
	\begin{equation*}
		u'(w)
		= u'\left( y \right)
		\exp\left( \int_y^w \frac{u''}{u'} \right) 
		\quad \text{and} \quad
		v'(w)
		= v'\left( y \right)
		\exp\left( \int_y^w \frac{v''}{v'} \right) 
	\end{equation*}
	%
	for any $y,w \in X$. Hence by the fundamental theorem of calculus, it holds for any $x<y<z$ in $X$ that
	%
	\begin{equation*}
		\frac{ u(z) - u(y) }{ u(y) - u(x) }
		= \frac
		{ \int_y^z
		\exp\left( \int_y^w \frac{u''}{u'} \right)
		\dd w }
		{ \int_x^y
		\exp\left( - \int_w^y \frac{u''}{u'} \right)
		\dd w } 
		\geq \frac
		{ \int_y^z
		\exp\left( \int_y^w \frac{v''}{v'} \right)
		\dd w }
		{ \int_x^y
		\exp\left( - \int_w^y \frac{v''}{v'} \right)
		\dd w } 
		= \frac{ v(z) - v(y) }{ v(y) - v(x) } .
	\end{equation*}
	%
\end{proof}

\begin{namedthm}[Pratt's theorem, part~3.]
	%
	\label{theorem:pratt_ce}
	%
	For a non-empty convex subset $X$ of $\R$ and continuous strictly increasing $u,v : X \to \R$, $u$ is less risk-averse than $v$ if and only if for every simple lottery $p \in \Delta^0(X)$, $u^{-1}\left( \int u \dd p \right) \geq v^{-1}\left( \int v \dd p \right)$.%
		\footnote{The equivalence of this condition with property~\ref{item:pratt:trans} in \hyperref[theorem:pratt]{part~1 of Pratt's theorem} was shown already by \textcite[][Theorem~92]{HardyLittlewoodPolya1934}, though without the expected-utility interpretation.}
	%
\end{namedthm}

\begin{exercise}[easy]
	%
	\label{exercise:pratt_ce_pf}
	%
	Prove it!
	%
\end{exercise}

The quantity $u^{-1}\left( \int u \dd p \right)$ is called the \emph{certainty equivalent} of the simple lottery $p \in \Delta^0(X)$. By construction, the decision-maker is indifferent between any lottery $p$ and its certainty equivalent.

\begin{exercise}
	%
	\label{exercise:background_risk}
	%
	Consider a decision-maker with expected-utility preferences over simple monetary lotteries $\Delta^0(\R)$, and let $v : \R \to \R$ denote her risk attitude. Suppose that the chosen lottery does not capture all risks borne by the decision-maker: her total take-home wealth is the sum of two random variables, namely the random draw from her chosen lottery and a second random variable capturing so-called \emph{background risk.} We assume that these two random variables are statistically independent. The decision-maker's valuation of any given lottery $p \in \Delta^0(\R)$ is then
	%
	\begin{equation*}
		\int \left[ \int v(x+w) \overline{p}(\dd w) \right] p(\dd x) ,
	\end{equation*}
	%
	where $\overline{p} \in \Delta^0(\R)$ is the distribution of the `background risk' random variable. That is, for any simple lotteries $p,q \in \Delta^0(\R)$, $p \succsim q$ if and only if 
	%
	\begin{equation*}
		\int \left[ \int v(x+w) \overline{p}(\dd w) \right] p(\dd x)
		\geq \int \left[ \int v(x+w) \overline{p}(\dd w) \right] q(\dd x) .
	\end{equation*}

	\begin{enumerate}[label=(\alph*)]
	
		\item Show that $\succsim$ is expected-utility. What is its risk attitude $u$?

		\item Find an example of simple lotteries $\overline{p},p,q \in \Delta^0(\R)$ and a risk attitude $v : \R \to \R$ such that $\int v \dd p < \int v \dd q$ but
		%
		\begin{equation*}
			\int \left[ \int v(x+w) \overline{p}(\dd w) \right] p(\dd x)
			> \int \left[ \int v(x+w) \overline{p}(\dd w) \right] q(\dd x) .
		\end{equation*}
		%
		(That is, the introduction of background risk leads to a choice reversal.)

		\item (hard) Find an example of simple lotteries $\overline{p},p \in \Delta^0(\R)$, an alternative $x \in \R$ and risk attitudes $v_1,v_2 : \R \to \R$ such that $v_1$ is strictly less risk-averse than $v_2$,%
			\footnote{That is, $v_1$ is less risk averse than $v_2$, and $v_2$ is not less risk-averse than $v_1$.}
		and yet there exist $p \in \Delta^0(\R)$ and $x \in \R$ such that
		%
		\begin{align*}
			\int v_1(x+w) \overline{p}(\dd w)
			&> \int \left[ \int v_1(y+w) \overline{p}(\dd w) \right] p(\dd y) 
			\quad \text{and}
			\\
			\int v_2(x+w) \overline{p}(\dd w)
			&< \int \left[ \int v_2(y+w) \overline{p}(\dd w) \right] p(\dd y) .
		\end{align*}
		%
		(That is, after the introduction of background risk, the behaviour of the decision-maker with risk attitude $v_1$ is no longer less risk-averse than that of the decision-maker with risk attitude $v_2$.)%
			\footnote{Solution: see \textcite{KihlstromRomerWilliams1981}.}
	
	\end{enumerate}
	%
\end{exercise}



%%%%%%%%%%%%%%%%%%%%%%%%%%%%%%%%%%%
%%%%%%%%%%%%%%%%%%%%%%%%%%%%%%%%%%%
\section{(Local) risk-neutrality}
\label{mone:abs}
%%%%%%%%%%%%%%%%%%%%%%%%%%%%%%%%%%%
%%%%%%%%%%%%%%%%%%%%%%%%%%%%%%%%%%%

When alternatives are monetary, there is a natural benchmark risk attitude: \emph{risk-neutrality,} meaning evaluating every lottery at its expectation.

\begin{definition}
	%
	\label{definition:risk_neutr}
	%
	Let $X$ be a non-empty subset of $\R$. A preference $\succsim$ on $\Delta^0(X)$ is called \emph{risk-neutral} iff for every simple lottery $p \in \Delta^0(X)$, $p \sim \int x p(\dd x)$.
	%
\end{definition}

In other words, a risk-neutral decision-maker is one who is always indifferent between receiving a lottery $p$ and receiving a sure payment equal to the expectation of $p$. Note that this definition makes sense (only) because $X \subseteq \R$; if $X$ were are arbitrary set (as in the previous chapter), then the expectation `$\int x p(\dd x)$' would be meaningless.

\begin{exercise}[easy]
	%
	\label{exercise:risk_neutr_eu}
	%
	Show the following.

	\begin{enumerate}[label=(\alph*)]
	
		\item There is only one risk-neutral preference: that is, if $\succsim$ and $\succsim'$ are both risk-neutral, then $\mathord{\succsim} = \mathord{\succsim'}$.

		\item The risk-neutral preference is expected-utility, with affine risk attitude.%
			\footnote{Don't get confused. A preference $\succsim$ is expected-utility iff it admits a utility representation $U : \Delta^0(X) \to \R$ that is affine, and this is equivalent to the existence of a risk attitude, i.e. a $u : X \to \R$ such that $U(p) = \int u \dd p$ for every $p \in \Delta^0(X)$. Expected utility does not restrict the shape of $u$. But risk-neutrality does: for a risk-neutral preference $\succsim$, the risk attitude $u$ is itself an affine function.}
	
	\end{enumerate}
	%
\end{exercise}

\begin{definition}
	%
	\label{definition:risk_av}
	%
	A preference $\succsim$ on $X \subseteq \R$ is called \emph{risk-averse (risk-seeking)} iff it is more (less) risk-averse than the risk-neutral preference.
	%
\end{definition}

Note that a risk-averse preference need not be expected-utility. For expected-utility preferences, \hyperref[theorem:pratt]{Pratt's theorem} delivers a characterisation of risk-aversion (and of risk-seeking, though we omit this):

\begin{corollary}
	%
	\label{corollary:pratt_abs}
	%
	For a non-empty convex subset $X$ of $\R$ and a function $u : X \to \R$, $u$ is risk-averse (i.e. $\int u \dd p \leq u\left( \int x p(\dd x) \right)$ for every $p \in \Delta^0(X)$) if and only if $u$ is concave and strictly increasing. If in addition $X$ is open and $u$ is twice continuously differentiable with $u'>0$, then $u$ is risk-averse if and only if $u''/u' \leq 0$.
	%
\end{corollary}

\begin{exercise}
	%
	\label{exercise:pratt_abs_pf}
	%
	Prove it!
	%
\end{exercise}

\begin{exercise}
	%
	\label{exercise:time}
	%
	Consider a decision-maker with a preference $\succsim$ over $\Delta^0(\R_+)$, where alternatives $x \in \R_+$ are interpreted as \emph{dates} rather than monetary amounts. In other words, each $p \in \Delta^0(X)$ is a probability distribution describing \emph{when} something will happen. Assume that $\succsim$ has the standard form assumed e.g. in macroeconomics: there is an $r>0$ such that for all simple lotteries $p,q \in \Delta^0(\R_+)$, $p \succsim q$ if and only if $\int e^{-rt} p(\dd t) \geq \int e^{-rt} q(\dd t)$. The interpretation is that the decision-maker earns a positive payoff (normalised to one) when the event takes place, and that these payoffs are discounted at rate $r$.

	\begin{enumerate}[label=(\alph*)]
	
		\item Is $\succsim$ expected-utility?

		\item Is $\succsim$ risk-averse? Risk-neutral? Risk-seeking?
	
	\end{enumerate}
	%
\end{exercise}

\begin{namedthm}[\Cref*{exercise:eu_choice} {\normalfont (continued from \cpageref{exercise:eu_choice,exercise:eu_choice_axioms})}.]
	%
	\label{exercise:eu_choice_riskseeking}
	%
	Let $X$ be a non-empty subset of $\R$, and fix a non-empty set $A$ and a collection $(u_a)_{a \in A}$ of maps $X \to \R$. Let $\succsim$ be the preference represented by $U^{(A,(u_a)_{a \in A})}$.

	\begin{enumerate}[label=(\alph*)]
	
		\item Remind yourself from earlier in this exercise (\cpageref{exercise:eu_choice}) that $\succsim$ is not generally expected-utility.

		\item Show that if for each $a \in A$, $u_a$ is convex, then $\succsim$ is risk-seeking.

		\item Show by example that it is possible for $u_a$ to be concave for each $a \in A$ without $\succsim$ being risk-averse.
	
	\end{enumerate}
	%
\end{namedthm}

Given a non-empty convex subset $X$ of $\R$, a simple lottery $p \in \Delta^0(X)$, an alternative $x \in X$ and a scalar $\lambda \in [0,1]$, let $p^\lambda x \in \Delta^0(X)$ denote the distribution of the random variable $\lambda \boldsymbol{X} + (1-\lambda) x$ when the random variable $\boldsymbol{X}$ is drawn from $p$, i.e.
%
\begin{multline*}
	\left( p^\lambda x \right)(y)
	= \PP\left( \lambda \boldsymbol{X} + (1-\lambda) x = y \right)
	\\
	= \PP\left( \boldsymbol{X} = \frac{ y - (1-\lambda) x }{ \lambda } \right)
	= p\left( \frac{ y - (1-\lambda) x }{ \lambda } \right) 
	\quad \text{for each $y \in X$.}
\end{multline*}
%
It's a mouthful, but all it says is that $p^\lambda x$ is the lottery which delivers $\lambda$ exposure to the lottery $p$ and $1-\lambda$ exposure to the sure thing $x$.

\begin{definition}
	%
	\label{definition:local_rn}
	%
	Fix a non-empty convex subset $X$ of $\R$. A preference $\succsim$ on $\Delta^0(X)$ is \emph{locally risk-neutral} iff for any simple lottery $p \in \Delta^0(X)$ and alternative $x \in X$ such that $\int y p(\dd y) > x$, it holds that $p^\lambda x \succ x$ for all sufficiently small $\lambda \in (0,1]$.
	%
\end{definition}

In words, a locally risk-neutral decision-maker is one who evaluates any risk $p \in \Delta^0(X)$ according to its expected value $\int y p(\dd y)$, so long as her exposure $\lambda \in (0,1]$ to this risk is small enough. In particular, whenever the expected value of $p$ exceeds $x$, she strictly prefers to move away from pure exposure to $x$ toward at least a little exposure to $p$.

The following result says, basically, that an expected-utility preference whose risk attitude is strictly increasing (she likes money) and differentiable must be locally risk-neutral. (I say `basically' because we strengthen `strictly increasing' to `strictly positive derivative'.)

\begin{proposition}[\cite{Arrow1965}]
	%
	\label{proposition:local_rn}
	%
	Let $X$ be a non-empty open convex subset of $\R$, and let $\succsim$ be a preference on $\Delta^0(X)$. If $\succsim$ is expected-utility with risk attitude that is differentiable with strictly positive derivative, then $\succsim$ is locally risk-neutral.
	%
\end{proposition}

The idea behind the proof is simply to recollect that a differentiable function is (by definition) precisely one which is everywhere `locally affine': precisely, $u : X \to \R$ is differentiable at $x \in X$ if and only if there exists an affine function $y \mapsto ay+b$ such that $u(y) - (ay+b) = o(y-x)$ for all $y \in X$.%
	\footnote{`Little o' notation works as follows: $f(\eps)=o(1)$ if and only if $f(\eps) \to 0$ as $\eps \to 0$, and $g(\eps)=o(h(\eps))$ if and only if $g(\eps)/h(\eps) = o(1)$.}
In particular, $a=u'(x)$ and $b=u(x)-u'(x)x$.

\begin{proof}
	%
	Let $X \subseteq \R$ be non-empty, open and convex, and let $\succsim$ be an expected-utility preference on $\Delta^0(X)$ with risk attitude $u : X \to \R$ which is continuously differentiable with $u'>0$. Fix a simple lottery $p \in \Delta^0(X)$ and an alternative $x \in X$ such that $\int y p(\dd y) > x$, and define $V : [0,1] \to \R$ by
	%
	\begin{equation*}
		V(\lambda)
		\coloneqq \int u \dd \left( p^\lambda x \right)
		= \int u( \lambda y + (1-\lambda) x ) p(\dd y) ;
	\end{equation*}
	%
	we will show that
	%
	\begin{equation*}
		\lim_{\lambda \searrow 0} \frac{V(\lambda)-V(0)}{\lambda} 
	\end{equation*}
	%
	(exists and) is strictly positive. This suffices since it implies that $\int u \dd \bigl( p^\lambda x \bigr) = V(\lambda) > V(0) = u(x)$ for all sufficiently small $\lambda>0$.

	To that end, note that for all $y \in X \setminus \{x\}$ and $\lambda \in (0,1]$,
	%
	\begin{equation*}
		\frac{ u( \lambda y + (1-\lambda) x ) - u(x) }{\lambda}
		= \frac{ u( \lambda y + (1-\lambda) x ) - u(x) }{ [ \lambda y + (1-\lambda) x ] - x } (y-x) .
	\end{equation*}
	%
	Since $u$ is differentiable, it follows that
	%
	\begin{equation*}
		\lim_{\lambda \searrow 0} \frac{ u( \lambda y + (1-\lambda) x ) - u(x) }{\lambda}
		= u'(x) (y-x) 
		\quad \text{for every $y \in X$.}
	\end{equation*}
	%
	Hence
	%
	\begin{multline*}
		\lim_{\lambda \searrow 0} \frac{V(\lambda)-V(0)}{\lambda}
		= \lim_{\lambda \searrow 0} \int \frac{ u( \lambda y + (1-\lambda) x ) - u(x) }{\lambda} p( \dd y )
		\\
		= \int u'(x) (y-x) p(\dd y)
		= u'(x) \left( \int y p(\dd y) - x \right)
		> 0 ,
	\end{multline*}
	%
	where the inequality holds since $u'>0$ and $\int y p(\dd y) > x$.
	%
\end{proof}



%%%%%%%%%%%%%%%%%%%%%%%%%%%%%%%%%%%
%%%%%%%%%%%%%%%%%%%%%%%%%%%%%%%%%%%
\section[Notions of `better' \emph{(not yet written)}]{Notions of `better'}
\label{mone:stoch_high}
%%%%%%%%%%%%%%%%%%%%%%%%%%%%%%%%%%%
%%%%%%%%%%%%%%%%%%%%%%%%%%%%%%%%%%%

\emph{See sections~4.1 and 4.2 in \textcite{Sarver2023}, and chapter~3 in \textcite{Liang2023}. For (encyclopædic) further reading, see chapter~1 (especially sections~1.A and 1.C) in \textcite{ShakedShanthikumar2007}. First-order stochastic dominance and its characterisations were introduced to economics by \textcite{HadarRussell1969,HanochLevy1969}, but are presumably older. Similarly for the likelihood ratio order, which I believe was introduced into economics by \textcite{Milgrom1981}.}

% {\color{blue}FOSD (equiv of utility-based, pointwise, embedding; see Sarver 4.2); MLRP (equiv of def'n, conditional FOSD on intervals)}



%%%%%%%%%%%%%%%%%%%%%%%%%%%%%%%%%%%
%%%%%%%%%%%%%%%%%%%%%%%%%%%%%%%%%%%
\section[Notions of `riskier' \emph{(not yet written)}]{Notions of `riskier'}
\label{mone:stoch_risk}
%%%%%%%%%%%%%%%%%%%%%%%%%%%%%%%%%%%
%%%%%%%%%%%%%%%%%%%%%%%%%%%%%%%%%%%

\emph{See section~4.3 in \textcite{Sarver2023}. For (encyclopædic) further reading, see chapter~3 (especially section~3.A) in \textcite{ShakedShanthikumar2007}. The convex order and some characterisations of it were introduced to economics by \textcite{RotschildStiglitz1970}, but can be traced back at least to \textcite{HardyLittlewoodPolya1934}.}

% {\color{blue}weak notion from before: less dispersed iff constant; convex order (equiv of convex order, pointwise integrals, embedding; see Sarver 4.3); Rotschild--Stiglitz 1976 / Ross 1981 / Diamond--Stiglitz; maybe Aumann--Serrano, Hart}




%%%%%%%%%%%%%%%%%%%%%%
%%%%%%%%%%%%%%%%%%%%%%
%%%%%%%%%%%%%%%%%%%%%%
\chapter{Ambiguity}
\label{ch_ambi}
%%%%%%%%%%%%%%%%%%%%%%
%%%%%%%%%%%%%%%%%%%%%%

% Copyright (c) 2025 Carl Martin Ludvig Sinander.

% This program is free software: you can redistribute it and/or modify
% it under the terms of the GNU General Public License as published by
% the Free Software Foundation, either version 3 of the License, or
% (at your option) any later version.

% This program is distributed in the hope that it will be useful,
% but WITHOUT ANY WARRANTY; without even the implied warranty of
% MERCHANTABILITY or FITNESS FOR A PARTICULAR PURPOSE. See the
% GNU General Public License for more details.

% You should have received a copy of the GNU General Public License
% along with this program. If not, see <https://www.gnu.org/licenses/>.

%%%%%%%%%%%%%%%%%%%%%%%%%%%%%%%%%%%%%%%%%%%%%%%%%%%%%%%%%%%%%%%%%%%%%%%

In this chapter, we study unquantifiable or `subjective' uncertainty: \emph{ambiguity.} In particular, unlike in \cref{ch_risk,ch_mone}, we do \emph{not} assume that for each uncertain prospect, the decision-maker has a probabilistic belief (`lottery') about how likely various payoff-relevant consequences are to arise; furthermore, even if she does have such a belief, we do not assume that we (the economic modellers) know what it is. Formally, we model prospects as \emph{acts,} meaning maps from states of the world to payoff-relevant consequences, and we study choice among (or preferences over) acts.



%%%%%%%%%%%%%%%%%%%%%%%%%%%%%%%%%%%
%%%%%%%%%%%%%%%%%%%%%%%%%%%%%%%%%%%
\section{Preferences over acts}
\label{sec:acts}
%%%%%%%%%%%%%%%%%%%%%%%%%%%%%%%%%%%
%%%%%%%%%%%%%%%%%%%%%%%%%%%%%%%%%%%

The \emph{Savage framework} (after \cite{Savage1954}, though really the framework predates him) is the following environment. There is a non-empty set $Z$ of \emph{consequences} (also called `prizes' or `outcomes'). There is also a non-empty finite set $S$ of states of the world. A \emph{(Savage) act} is a map $f : S \to Z$, i.e. an element of $Z^S$. By a standard abuse, the act in $Z^S$ that is constant at $z \in Z$ is denoted simply `$z$'. A \emph{preference} is a complete and transitive binary relation on the set $Z^S$ of all acts.

The idea is that what the decision-maker actually chooses among (i.e. has preferences over) is prospects, that what she actually cares about is consequences, and that she is uncertain about which prospects lead to which consequences. Prospects are modelled as acts, which deliver a state-contingent consequence. The state of the world should be thought of as a summary of all relevant facts which the decision-maker does not know; in particular, it contains all the information required to determine, for each prospect, which consequence will be delivered.

`Subjective expected-utility' preferences are those which can be viewed as arising from maximisation of the expectation of some function $u : Z \to \R$ under some probability $\mu$ on $S$. By `probability on $S$', I mean a function $\mu : S \to [0,1]$ such that $\sum_{s \in S} \mu(s) = 1$.

\begin{definition}[\cite{Ramsey1926}]
	%
	\label{definition:seu_savage}
	%
	Consider the Savage framework with states $S$ and consequences $Z$, and fix a preference $\succsim$. Given a map $u : Z \to \R$ and a probability $\mu$ on $S$, we say that the pair $(u,\mu)$ is a \emph{subjective expected-utility representation} of $\succsim$ if and only if for any (Savage) acts $f,g : S \to Z$, $f \succsim g$ if and only if $\sum_{s \in S} u(f(s)) \mu(s) \geq \sum_{s \in S} u(g(s)) \mu(s)$.
	%
\end{definition}

The function $u : Z \to \R$ is called a \emph{risk attitude} (or `vNM utility function', or `Bernoulli utility function'). The probability $\mu : S \to [0,1]$ is called a \emph{(subjective) belief}.



%%%%%%%%%%%%%%%%%%%%%%%%%%%%%%%%%%%
\subsection{The Anscombe--Aumann framework}
\label{sec:acts:aa}
%%%%%%%%%%%%%%%%%%%%%%%%%%%%%%%%%%%

Imagine in addition that there is a non-empty finite set $X$ of `alternatives', with generic elements $x,y,z,w \in X$. A \emph{lottery} is (a probability mass function of) a probability distribution over $X$: formally, a function $p : X \to [0,1]$ such that $\sum_{x \in X} p(x) = 1$. We write $\Delta(X)$ for the set of all lotteries, with generic elements $p,q,r \in \Delta(X)$. By the familiar abuse, the lottery in $\Delta(X)$ that is degenerate at $x \in X$ is denoted simply `$x$'.

The \emph{Anscombe--Aumann framework} \parencite[after][]{AnscombeAumann1963} is the special case of the Savage framework in which it is assumed that there exists a non-empty finite set $X$ of alternatives such that $Z = \Delta(X)$. That is, each consequence is a lottery over a set $X$ of underlying payoff-relevant alternatives (and, conversely, all such lotteries are consequences). In this special case, acts are maps $S \to \Delta(X)$, and are sometimes called `Anscombe--Aumann acts' to disambiguate.

Another way of thinking about the Anscombe--Aumann framework is to imagine starting with the Savage framework with consequences $Z=X$, and then enriching it by allowing for more acts, in particular allowing not only for acts that deliver a sure alternative in each state, but also acts that deliver lotteries over alternatives. Same thing, just a slightly different perspective.

Whatever perspective we adopt, the assumption that we make when moving from the Savage to the Anscombe--Aumann framework is precisely that it is possible in principle for us (as economic modellers) to observe how the decision-maker would choose among acts which deliver a state-contingent \emph{lottery} over underlying payoff-relevant alternatives. This seems a completely reasonable assumption for the purposes of economic modelling; in a laboratory, you would achieve this by flipping coins (or promising to). Nothing funny here.%
	\footnote{However, some people are interested in the Savage framework for quasi-philosophical rather than economic-modelling reasons; in particular, some believe that the Savage framework holds answers to questions like `what is the true nature of ``probability''?' For answering questions like that, the Anscombe--Aumann framework is certainly unsatisfactory, since something called `probability' is part of the description of the model!}

\begin{definition}[\cite{AnscombeAumann1963}]
	%
	\label{definition:seu_aa}
	%
	Consider the Anscombe--Aumann framework with states $S$ and alternatives $X$, and fix a preference $\succsim$. Given a map $u : X \to \R$ and a probability $\mu$ on $S$, we say that the pair $(u,\mu)$ is a \emph{subjective expected-utility representation} of $\succsim$ if and only if for any (Anscombe--Aumann) acts $f,g : S \to \Delta(X)$, $f \succsim g$ if and only if $\sum_{s \in S} ( \int u \dd [f(s)] ) \mu(s) \geq \sum_{s \in S} ( \int u \dd [g(s)] ) \mu(s)$. (Given a lottery $p \in \Delta(X)$, `$\int u \dd p$' is shorthand for $\sum_{x \in X} p(x) u(x)$.)
	%
\end{definition}

Note that \Cref{definition:seu_aa} is more demanding that \Cref{definition:seu_savage}: like \Cref{definition:seu_savage}, it demands that the uncertainty about the state be evaluated by taking an expectation (under a `subjective' probability $\mu$), but it additionally demands that the uncertainty about which alternative a given lottery will deliver also be evaluated by taking an expectation. Here is another way of saying the same thing: $(u,\mu)$ is a subjective expected-utility representation of $\succsim$ in the sense of \Cref{definition:seu_aa} if and only if $(U,\mu)$ is a subjective expected-utility representation of $\succsim$ in the sense of \Cref{definition:seu_savage}, where $U : \Delta(X) \to \R$ is the affine function given by $U(p) \coloneqq \int u \dd p$ for every $p \in \Delta(X)$.

The value of the Anscombe--Aumann special case of the Savage framework is that it is far more tractable, as we shall see.

Preferences that don't admit a subjective expected-utility representation arise naturally in many contexts. Psychological reasons for this are often emphasised, e.g. the enormous literature about the \textcite{Ellsberg1961} thought experiment. But straightforwardly economic forces also frequently produce behaviour that is inconsistent with subjective expected utility. There are many examples, of which perhaps the most `economic' of all is the following.

\begin{exercise}[compare with \Cref{exercise:eu_choice} in \cref{ch_risk}, \cpageref{exercise:eu_choice}]
	%
	\label{exercise:mh}
	%
	Consider a decision-maker who must not only choose an Anscombe--Aumann act, but must also choose an action. Imagine, for example, a manager who chooses among projects (risky prospects, modelled as acts) and, after choosing her project, chooses how to operate the project, e.g. what staff to employ on her team and how to organise them. The operational options are modelled as a non-empty set $A$ of actions. Suppose that for each given action $a \in A$, the decision-maker has subjective expected-utility preferences: she evaluates each act $f : S \to \Delta(X)$ at $\sum_{s \in S} ( \int u_a \dd [f(s)] ) \mu(s)$, for some risk attitude $u_a : X \to \R$ and belief $\mu : S \to [0,1]$. Then taking into account optimal action choice, she evaluates each act $f : S \to \Delta(X)$ at $U^{(A,(u_a)_{a \in A},\mu)}(f) \coloneqq \max_{a \in A} \sum_{s \in S} ( \int u_a \dd [f(s)] ) \mu(s)$. (That is: her preference $\succsim$ is such that for any acts $f,g : S \to \Delta(X)$, $f \succsim g$ holds if and only if $U^{(A,(u_a)_{a \in A},\mu)}(f) \geq U^{(A,(u_a)_{a \in A},\mu)}(g)$.)

	\begin{enumerate}[label=(\alph*)]
	
		\item Remind yourself of \hyperref[exercise:mix_acts_affine]{\Cref*{exercise:mix_acts}} (\cref{ch_mix}, \cpageref{exercise:mix_acts_affine}), which says that a preference admits a subjective expected-utility representation if and only if it admits a utility representation $U : \Delta(X)^S \to \R$ that is affine.

		\item Show that $U^{(A,(u_a)_{a \in A},\mu)}$ is convex.%
			\footnote{There is a converse along the lines of \cref{footnote:convex_max} in \cref{ch_risk} (\cpageref{footnote:convex_max}).}

		\item Under what conditions is $U^{(A,(u_a)_{a \in A},\mu)}$ affine?
	
	\end{enumerate}
	%
\end{exercise}



%%%%%%%%%%%%%%%%%%%%%%%%%%%%%%%%%%%
%%%%%%%%%%%%%%%%%%%%%%%%%%%%%%%%%%%
\section{The Anscombe--Aumann theorem}
\label{sec:aa}
%%%%%%%%%%%%%%%%%%%%%%%%%%%%%%%%%%%
%%%%%%%%%%%%%%%%%%%%%%r%%%%%%%%%%%%%

\emph{This section draws on \textcite[chapter~7]{Kreps1988}.}

Which preferences over Anscombe--Aumann acts admit a subjective expected-utility representation?

\begin{definition}
	%
	\label{definition:aa_indep}
	%
	In the Anscombe--Aumann framework with states $S$ and alternatives $X$, a preference $\succsim$ satisfies \emph{independence} iff for all (Anscombe--Aumann) acts $f,g,h : S \to \Delta(X)$ and all $\alpha \in [0,1]$, $f \sim g$ implies $\alpha f + (1-\alpha) h \sim \alpha g + (1-\alpha) h$.
	%
\end{definition}

\begin{definition}
	%
	\label{definition:aa_continuity}
	%
	In the Anscombe--Aumann framework with states $S$ and alternatives $X$, a preference $\succsim$ satisfies \emph{mixture continuity} iff for all acts $f,g,h : S \to \Delta(X)$ such that $f \succ g \succ h$, the sets $\{ \alpha \in [0,1] : \alpha f + (1-\alpha) h \succsim g \}$ and $\{ \alpha \in [0,1] : \alpha f + (1-\alpha) h \precsim g \}$ are closed in $[0,1]$.
	%
\end{definition}

Independence and mixture continuity are plainly exactly the concepts bearing those names in the general mixture-space context of \cref{ch_mix}, specialised to the particular mixture space $\Delta(X)^S$ equipped with the mixture operation $(f,\alpha,g) \mapsto \alpha f + (1-\alpha) g$.

To state the next property, we need a piece of notation: for an act $f : S \to \Delta(X)$, a lottery $p \in \Delta(X)$, and a state $s \in S$, we write $f_s p : S \to \Delta(X)$ for the act given by
%
\begin{equation*}
	(f_s p)(t) \coloneqq
	\begin{cases}
		p & \text{if $t=s$} \\
		f(t) & \text{otherwise.}
	\end{cases}
\end{equation*}

\begin{definition}
	%
	\label{definition:aa_sub}
	%
	In the Anscombe--Aumann framework with states $S$ and alternatives $X$, a preference $\succsim$ satisfies \emph{state-separability} iff for any act $f : S \to \Delta(X)$, any lotteries $p,q \in \Delta(X)$ and any states $s,t \in S$, $f_s p \succsim f_s q$ implies $f_t p \succsim f_t q$.
	%
\end{definition}

(How would you interpret state-separability?)

\begin{definition}
	%
	\label{definition:aa_nondegen}
	%
	In the Anscombe--Aumann framework with states $S$ and alternatives $X$, a preference $\succsim$ satisfies \emph{non-degeneracy} iff there exist $f,g : S \to \Delta(X)$ such that $f \succ g$.
	%
\end{definition}

Say that a subjective expected-utility representation $(u,\mu)$ is non-de\-ge\-ne\-rate if and only if $u$ is non-constant.

\begin{namedthm}[Anscombe--Aumann theorem {\normalfont \parencite{AnscombeAumann1963}}.]
	%
	\label{theorem:aa}
	%
	Consider the Anscombe--Aumann framework with states $S$ and alternatives $X$, and let $\succsim$ be a preference. $\succsim$ admits a non-degenerate subjective expected-utility representation if and only if it satisfies independence, mixture continuity, state-separability, and non-degeneracy. Furthermore, if $(u,\mu)$ and $(v,\nu)$ are both subjective expected-utility representations of $\succsim$, then $\mu=\nu$, and there exist $a>0$ and $b \in \R$ such that $u = a v + b$.
	%
\end{namedthm}

In other words, independence, mixture continuity, state-separability, and non-degeneracy together characterise non-degenerate subjective expected-utility preferences, the belief is unique, and the risk attitude is unique up to positive affine transformations.

The theorem remains true if the axioms are modified in various small ways: independence can be weakened or altered as in the mixture-space theorem, mixture continuity can be replaced with any one of a number of alternative notions of `continuity', and state-separability can also be replaced.%
	\footnote{Non-degeneracy is very weak, but dropping it does have some consequences. You can work these out for yourself if you're interested.}
In fact, the most commonly-stated version of the \hyperref[theorem:aa]{Anscombe--Aumann theorem} features a `monotonicity' property in place of state-separability;%
	\footnote{A preference $\succsim$ satisfies \emph{monotonicity} iff for all $f,g : S \to \Delta(X)$, if $f(s) \succsim g(s)$ for every $s \in S$, then $f \succsim g$.}
the version above with state-separability is from \textcite{Kreps1988}.

\begin{exercise}
	%
	\label{exercise:aa_easy}
	%
	Prove the `only if' part of the first claim in the \hyperref[theorem:aa]{Anscombe--Aumann theorem} (namely, that a preference which admits a non-degenerate subjective expected-utility representation must satisfy independence, mixture continuity, state-separability, and non-degeneracy).
	%
\end{exercise}

\begin{proof}[Proof of the `if' part of the first claim in the {\hyperref[theorem:aa]{Anscombe--Aumann theorem}}]
	%
	Let $\succsim$ satisfy independence, mixture continuity, state-separability, and non-degeneracy; we will show that it admits a non-degenerate subjective expected-utility representation $(u,\mu)$.

	Since $\succsim$ satisfies independence and mixture continuity, the \hyperref[theorem:mix]{mixture-space theorem} (\cref{ch_mix}, \cpageref{theorem:mix}) implies that there exists an affine $U : \Delta(X)^S \to \R$ that represents $\succsim$. By \hyperref[exercise:mix_acts_affine]{\Cref*{exercise:mix_acts}} (\cpageref{exercise:mix_acts_affine}), it follows that there exists a collection $(u_s)_{s \in S}$ of functions $X \to \R$ such that $U(f) = \sum_{s \in S} \int u_s \dd[f(s)]$ for each act $f : S \to \Delta(X)$.

	Call a state $s \in S$ \emph{non-null} iff there exist an act $f : S \to \Delta(X)$ and a lottery $p \in \Delta(X)$ such that $f_s p \succ f$. (In other words, the decision-maker cares what happens in state $s$.) It is easy to see that for each state $s \in S$, $u_s$ is non-constant if and only if $s$ is non-null. By non-degeneracy, there must be at least one non-null state; let $s^\star \in S$ be one such.

	Since $\succsim$ is represented by $U$ and $U(g) = \sum_{t \in S} \int u_t \dd[g(t)]$ for each act $g : S \to \Delta(X)$, it holds for any act $f : S \to \Delta(X)$, any lotteries $p,q \in \Delta(X)$ and any non-null state $s \in S$ that
	%
	\begin{align*}
		&\int u_s \dd p \geq \int u_s \dd q
		\\
		\text{iff} \quad
		&\sum_{t \in S} \int u_t \dd [f_s p] \geq \sum_{t \in S} \int u_t \dd [f_s q]
		\\
		\text{iff} \quad
		&f_s p \succsim f_s q
		\\
		\text{iff} \quad
		&f_{s^\star} p \succsim f_{s^\star} q
		\\
		\text{iff} \quad
		&\sum_{t \in S} \int u_t \dd [f_{s^\star} p] \geq \sum_{t \in S} \int u_t \dd [f_{s^\star} q]
		\\
		\text{iff} \quad
		&\int u_{s^\star} \dd p \geq \int u_{s^\star} \dd q
	\end{align*}
	%
	where the third `iff' holds by state-separability. This shows that $u_s$ and $u_{s^\star}$ represent the same preference over lotteries $\Delta(X)$. Hence by the \hyperref[theorem:vNM]{von Neumann--Morgenstern theorem} (\cref{ch_risk}, \cpageref{theorem:vNM}), there exist $a_s > 0$ and $b_s \in \R$ such that $u_s = a_s u_{s^\star} + b_s$. For any null state $s \in S$, since $u_s$ is constant, we have $u_s = a_s u_{s^\star} + b_s$ where $a_s=0$ and $b_s \in \R$. Let $a_{s^\star} \coloneqq 1$ and $b_{s^\star} \coloneqq 0$.

	Let $a \coloneqq \sum_{s \in S} a_s$ and $b \coloneqq \sum_{s \in S} b_s$. Let $u \coloneqq a u_{s^\star} + b$, and define $\mu : S \to \R$ by $\mu(s) \coloneqq a_s/a$ for each $s \in S$. Then $\mu$ is a probability on $S$, $u$ is non-constant since $s^\star$ is non-null, and for each act $f : S \to \Delta(X)$,
	%
	\begin{multline*}
		U(f)
		= \sum_{s \in S} \int u_s \dd[f(s)]
		= \sum_{s \in S} \int (a_s u_{s^\star} + b_s) \dd[f(s)]
		\\
		= a \left[ \sum_{s \in S} \left( \int  u_{s^\star} \dd[f(s)] \right) \frac{a_s}{a} \right] + b
		= \sum_{s \in S} \left( \int u \dd[f(s)] \right) \mu(s) .
	\end{multline*}
	%
	Hence $(u,\mu)$ is a non-degenerate subjective expected-utility representation of $\succsim$.
	%
\end{proof}

\begin{exercise}
	%
	\label{exercise:aa_uniqueness}
	%
	Prove the second (`furthermore') claim in the \hyperref[theorem:aa]{Anscombe--Aumann theorem} (namely, the uniqueness of the belief and the uniqueness up to positive affine transformations of the risk attitude).
	%
\end{exercise}

\begin{namedthm}[\Cref*{exercise:mh} {\normalfont (continued; based on \cite{Sinander2025})}.]
	%
	\label{exercise:mh_axioms}
	%
	Read \textcite{MaccheroniMarinacciRustichini2006}. Characterise the behavioural content of the subjective-expected-utility-with-choice model: that is, identify a set of properties such that a preference $\succsim$ satisfies these properties if and only if there exists a non-empty set $A$, a collection $(u_a)_{a \in A}$ of maps $X \to \R$, and a probability $\mu$ on $S$ such that $U^{(A,(u_a)_{a \in A},\mu)}$ represents $\succsim$ (that is, for any acts $f,g : S \to \Delta(X)$, $f \succsim g$ holds if and only if $U^{(A,(u_a)_{a \in A},\mu)}(f) \geq U^{(A,(u_a)_{a \in A},\mu)}(g)$).
	%
\end{namedthm}



%%%%%%%%%%%%%%%%%%%%%%%%%%%%%%%%%%%
%%%%%%%%%%%%%%%%%%%%%%%%%%%%%%%%%%%
\section[Savage's theorem \emph{(not yet written)}]{Savage's theorem}
\label{ambi:savage}
%%%%%%%%%%%%%%%%%%%%%%%%%%%%%%%%%%%
%%%%%%%%%%%%%%%%%%%%%%%%%%%%%%%%%%%

\emph{See chapters~8 and 9 in \textcite{Kreps1988} or section~8.2 in \textcite{Strzalecki2023}. The theorem is due to \textcite{Savage1954}.}




%%%%%%%%%%%%%%%%%%%%%%
%%%%%%%%%%%%%%%%%%%%%%
%%%%%%%%%%%%%%%%%%%%%%
\chapter{Information}
\label{ch_info}
%%%%%%%%%%%%%%%%%%%%%%
%%%%%%%%%%%%%%%%%%%%%%
%%%%%%%%%%%%%%%%%%%%%%

% Copyright (c) 2025 Carl Martin Ludvig Sinander.

% This program is free software: you can redistribute it and/or modify
% it under the terms of the GNU General Public License as published by
% the Free Software Foundation, either version 3 of the License, or
% (at your option) any later version.

% This program is distributed in the hope that it will be useful,
% but WITHOUT ANY WARRANTY; without even the implied warranty of
% MERCHANTABILITY or FITNESS FOR A PARTICULAR PURPOSE. See the
% GNU General Public License for more details.

% You should have received a copy of the GNU General Public License
% along with this program. If not, see <https://www.gnu.org/licenses/>.

%%%%%%%%%%%%%%%%%%%%%%%%%%%%%%%%%%%%%%%%%%%%%%%%%%%%%%%%%%%%%%%%%%%%%%%

In this chapter, we study information and its value. We employ the Savage framework from the previous chapter, in which a decision-maker is uncertain about some payoff-relevant facts, modelled as a `state (of the world)'. Learning is modelled as obtaining information about which state prevails, as described in \cref{info:info} below. In \cref{info:split}, we study how information may be represented by a distribution of (posterior) beliefs. In \cref{info:blackwell}, we study the (instrumental) value of information, rooted in how information changes the decision-maker's choice from among a set of acts available to her.

Relative to the previous chapter, we slightly change notation by writing $\Theta$ (rather than $S$) for the set of states of the world (with typical elements $\theta,\theta',\theta'' \in \Theta$), and by writing $p,q,r \in \Delta(\Theta)$ for beliefs (probabilities on $\Theta$).



%%%%%%%%%%%%%%%%%%%%%%%%%%%%%%%%%%%
%%%%%%%%%%%%%%%%%%%%%%%%%%%%%%%%%%%
\section{Beliefs and information}
\label{info:info}
%%%%%%%%%%%%%%%%%%%%%%%%%%%%%%%%%%%
%%%%%%%%%%%%%%%%%%%%%%%%%%%%%%%%%%%

\emph{This text of this section draws somewhat on \textcite{disc}.}

There is a non-empty finite set $\Theta$ of states of the world. The state $\theta \in \Theta$ summarises all payoff-relevant matters of fact; in particular, it fully pins down the payoff consequences of any prospect available to the decision-maker. (We will study choice among prospects/acts in \cref{info:blackwell} below.)

The decision-maker does not know which state $\theta \in \Theta$ prevails. We assume that she has a probabilistic belief $p \in \Delta(\Theta)$ about this,%
	\footnote{This will be the case if, for example, she has subjective expected-utility preferences over acts, as in the previous chapter; but again, we will not talk about choice among acts until \cref{info:blackwell}.}
where $\Delta(\Theta)$ denotes the set of all probabilities on $\Theta$ (functions $p : \Theta \to [0,1]$ such that $\sum_{\theta \in \Theta} p(\theta) = 1$). We assume that $p$ has \emph{full support}, i.e. belongs to $\interior(\Delta(\Theta))$, the interior of the set $\Delta(\Theta)$.%
	\footnote{That is, $p(\theta)>0$ for every $\theta \in \Theta$. This assumption is without loss of generality, because if there were a state $\theta \in \Theta$ with $p(\theta)=0$, then we could neglect $\theta$ entirely, by deleting it from $\Theta$.}
The distribution $p$ is called the decision-maker's \emph{prior belief} (or simply `prior'). The word `prior' is meant to emphasise that this is the decision-maker's belief \emph{before} she receives information.

Information is modelled as follows. A \emph{signal structure} is a pair $\langle S, \pi \rangle $, where $S$ is a non-empty finite set and $\pi : S \times \Theta \to [0,1]$ satisfies $\sum_{s \in S} \pi(s|\theta) = 1$ for each state $\theta \in \Theta$.%
	\footnote{Signal structures are also called `information structures or `(Blackwell) experiments.'}
The interpretation is that $S$ is the set of possible signals $s$, and that $\pi(s|\theta)$ is the probability that signal $s$ will be observed conditional on the state being $\theta$. We assume that for every signal $s \in S$, there is at least one state $\theta \in \Theta$ such that $\pi(s|\theta)>0$.%
	\footnote{This assumption is without loss of generality, because if there were a signal $s \in S$ with $\pi(s|\theta)=0$ for every $\theta \in \Theta$, then we could neglect $s$ entirely, by deleting it from $S$.}

\begin{remark}
	%
	\label{remark:rv_embedding}
	%
	You can (indeed, should!) think of a signal structure as a conditional probability distribution. In particular, you can think of the state as a finite-support random variable (defined on some probability space) with law $p$, and of each signal as another random variable (defined on that same probability space) that is jointly distributed with the state, with conditional distribution $\pi$. The joint distribution of the state and signal is $(\theta,s) \mapsto p(\theta) \pi(s|\theta)$. Everything said in this chapter could equivalently be said in terms of random variables rather than the distributions $p$ and $\pi$ (but it is usually more convenient to work with $p$ and $\pi$).%
		\footnote{Some authors identify a \emph{third} way of modelling information, as a finite partition of $\Theta \times [0,1]$. (This formalism is from \textcite{GreenStokey1978}.) But this is \emph{not} a third way; rather, it is a special case of the `jointly distributed random variables' formalism, in which the underlying probability space is a particular one, namely $\Theta \times [0,1]$ equipped with (the product $\sigma$-algebra inherited from the discrete $\sigma$-algebra on $\Theta$ and the Lebesgue $\sigma$-algebra on $[0,1]$, and) a probability measure $\mu$ (on that $\sigma$-algebra) whose marginal on $[0,1]$ equals the Lebesgue measure $\lambda$, i.e. $\mu(\Theta \times A) = \lambda(A)$ for every Lebesgue-measurable $A \subseteq [0,1]$. The interpretation of a finite partition $\mathcal{P}$ of $\Theta \times [0,1]$ is that a state--number pair $(\theta,i)$ is drawn from $\mu$, and the decision-maker observes (only) to which cell $P \in \mathcal{P}$ the pair $(\theta,i)$ belongs; equivalently, the decision-maker observes (only) the realisation of the cell-valued finite-support random variable $Y$ such that for each $(\theta,i) \in \Theta \times [0,1]$, $Y(\theta,i)$ is the unique $P \in \mathcal{P}$ to which $(\theta,i)$ belongs. (If you prefer, you can assign distinct numerical values to the cells via an arbitrary one-to-one map $f : \mathcal{P} \to \N$, and instead consider the integer-valued finite-support random variable $Z$ such that for each $(\theta,i) \in \Theta \times [0,1]$, $Z=f(P)$ where $P$ is the unique $Q \in \mathcal{P}$ to which $(\theta,i)$ belongs.) This particular probability space is fine for most purposes, but there is nothing special about it.}
	%
\end{remark}

So, the decision-maker has a prior belief $p \in \interior(\Delta(\Theta))$ and will observe some additional information as described by a signal structure $\langle S, \pi \rangle $. In particular, she does not observe the true state $\theta \in \Theta$, but she does observe the realised signal $s \in S$, which is drawn from the probability distribution $\pi(\cdot|\theta)$ (where, again, $\theta$ is the unknown true state). To estimate the state based on the observed signal $s \in S$, the decision-maker applies Bayes's rule. Concretely, the posterior probability which she assigns to the state being $\theta \in \Theta$, given that she observed signal $s \in S$, is
%
\begin{equation*}
	p_{\langle S, \pi \rangle }(\theta|s)
	\coloneqq \frac{p(\theta) \pi(s|\theta)}{\sum_{\theta' \in \Theta} p(\theta') \pi(s|\theta')} .
\end{equation*}

\begin{remark}
	%
	\label{remark:pop}
	%
	An alternative interpretation of this model is that the `distribution' (of states and signals) is a cross-sectional distribution in a population, rather than a probability distribution reflecting uncertainty. See e.g. \textcite{disc}.
	%
\end{remark}



%%%%%%%%%%%%%%%%%%%%%%%%%%%%%%%%%%%
%%%%%%%%%%%%%%%%%%%%%%%%%%%%%%%%%%%
\section{The splitting lemma}
\label{info:split}
%%%%%%%%%%%%%%%%%%%%%%%%%%%%%%%%%%%
%%%%%%%%%%%%%%%%%%%%%%%%%%%%%%%%%%%

Let $\Delta^0(\Delta(\Theta))$ denote the set of all finite-support probability distributions (or `simple lotteries') on $\Delta(\Theta)$; that is, all functions $\tau : \Delta(\Theta) \to [0,1]$ such that $\supp(\tau) \coloneqq \{ q \in \Delta(\Theta) : \tau(q) > 0 \}$ is finite and $\sum_{q \in \supp(\tau)} \tau(q) = 1$. We interpret each $\tau \in \Delta^0(\Delta(\Theta))$ as a \emph{distribution of posterior beliefs,} i.e. the distribution of a $\Delta(\Theta)$-valued random variable. (You may prefer to say `random vector', or `random function', or indeed `random belief'.)

Each prior belief $p \in \interior(\Delta(\Theta))$ and signal structure $\langle S, \pi \rangle $ together \emph{induce} a distribution of posterior beliefs, namely $\tau \in \Delta^0(\Delta(\Theta))$ given by
%
\begin{equation*}
	\tau(q)
	\coloneqq \sum_{\substack{(\theta,s) \in \Theta \times S :\\ p_{\langle S, \pi \rangle }(\cdot|s) = q}} p(\theta) \pi(s|\theta)
	\quad \text{for each $q \in \Delta(\Theta)$.}
\end{equation*}
%
This is simply the total probability (according to the joint distribution $(\theta,s) \mapsto p(\theta) \pi(s|\theta)$ of the state and signal) that the state and signal are drawn in such a way that the posterior belief is equal to $q$.

The following result characterises inducible distributions of posterior beliefs.

\begin{namedthm}[Splitting lemma {\normalfont \parencite{Blackwell1951}}.\footnotemark]
	%
	\label{lemma:split}
	%
	\footnotetext{A version of this result appears already in \textcite{HardyLittlewoodPolya1934}. In the literature, you will often see the \hyperref[lemma:split]{splitting lemma} attributed either to \textcite{AumannMaschler1995} or to \textcite{KamenicaGentzkow2011}.}
	Fix a non-empty finite set $\Theta$. For a (prior) belief $p \in \interior(\Delta(\Theta))$ and a finite-support distribution $\tau \in \Delta^0(\Delta(\Theta))$ of (posterior) beliefs, the following are equivalent:

	\begin{enumerate}[label=(\alph*)]
	
		\item \label{split:induc} There exists a signal structure $\langle S, \pi \rangle $ such that $p$ and $\langle S, \pi \rangle $ together induce $\tau$.

		\item \label{split:bary} $\int q \tau( \dd q ) = p$.
	
	\end{enumerate}
	%
\end{namedthm}

As in previous chapters, `$\int q \tau( \dd q )$' is shorthand for $\sum_{q \in \supp(\tau)} q \tau(q)$. And this object (i.e. the function $\theta \mapsto \sum_{q \in \supp(\tau)} q(\theta) \tau(q)$) is an element of $\Delta(X)$.

Property~\ref{split:induc} says that given the prior belief $p \in \interior(\Delta(\Theta))$, it is possible to induce the posterior-belief distribution $\tau \in \Delta^0(\Delta(X))$, by judiciously designing a signal structure $\langle S, \pi \rangle $. In short, it says that $\tau$ is `inducible' when the prior belief is $p$.

Property~\ref{split:bary} says that the mean (or `average' or `expectation') of the distribution $\tau$ is equal to $p$. Since $\tau$ is a distribution over vectors (or, if you prefer, functions), this equality is really a finite collection of (moment) equalities, viz. $\int q(\theta) \tau( \dd q ) = p(\theta)$ for each $\theta \in \Theta$. A `vector mean' of this sort is sometimes called a \emph{barycentre;} property~\ref{split:bary} says $\tau$ has barycentre $p$.

Another way of talking about property~\ref{split:bary} is in terms of belief `splits'. We can think of each $\tau \in \Delta^0(\Delta(X))$ that satisfies property~\ref{split:bary} as a `split' of the prior belief $p$, whereby probability mass is taken from $p$ and distributed `outwards' in such a way that the mean (or barycentre) remains equal to $p$. In this language, the \hyperref[lemma:split]{splitting lemma} says that all and only mean-preserving belief splits are inducible.

Yet another language for property~\ref{split:bary} is that of martingales. In particular, property~\ref{split:bary} holds if and only if for any $\Delta(\Theta)$-valued random variables $\boldsymbol{p}$ and $\boldsymbol{q})$ such that $\boldsymbol{p}=p$ a.s. and $\boldsymbol{q} \sim \tau$, the pair $(\boldsymbol{p},\boldsymbol{q})$ constitutes a (two-period, $\Delta(\Theta)$-valued) martingale.%
	\footnote{Given $N \in \N \cup \{+\infty\}$ and a non-empty convex subset $Y$ of a vector space, a sequence $(\boldsymbol{y}_n)_{n=1}^N$ of $Y$-valued random variables constitutes an ($N$-period, $Y$-valued) \emph{martingale} iff $\E( \boldsymbol{y}_{n+1} | \boldsymbol{y}_n ) = \boldsymbol{y}_n$ a.s. for every $n \in \{1,2,\dots,N-1\}$.}
The interpretation of $(\boldsymbol{p},\boldsymbol{q})$ is that it is the stochastic process representing the decision-maker's belief as it (randomly) evolves over time: her prior belief (before information arrived) is $\boldsymbol{p}$ ($=p$ a.s.), and her posterior belief (after information arrived) is $\boldsymbol{q}$. For this reason, property~\ref{split:bary} is often called the `martingale property (of beliefs)'.

A fourth and final name for property~\ref{split:bary} is `Bayes plausibility', as in `$\tau$ is Bayes plausible (given $p$)'.

\begin{proof}[Proof of the {\hyperref[lemma:split]{splitting lemma}}]
	%
	Let $\Theta$ be non-empty and finite. To show that \ref{split:induc} implies \ref{split:bary}, fix $p \in \interior(\Delta(\Theta))$ and $\tau \in \Delta^0(\Delta(\Theta))$, and suppose that there exists a signal structure $\langle S, \pi \rangle $ such that $p$ and $\langle S, \pi \rangle $ together induce $\tau$. Then for each $\theta'' \in \Theta$,
	%
	\begin{align*}
		\int q(\theta'') \tau( \dd q )
		&= \sum_{\theta \in \Theta} \sum_{s \in S} p_{\langle S, \pi \rangle }(\theta''|s) p(\theta) \pi( s | \theta )
		\\
		&= \sum_{\theta \in \Theta} \sum_{s \in S} \frac{p(\theta'') \pi(s|\theta'')}{\sum_{\theta' \in \Theta} p(\theta') \pi(s|\theta')} p(\theta) \pi( s | \theta )
		\\
		&= \sum_{s \in S} \sum_{\theta \in \Theta} \frac{p(\theta) \pi(s|\theta)}{\sum_{\theta' \in \Theta} p(\theta') \pi(s|\theta')} p(\theta'') \pi( s | \theta'' )
		\\
		&= p(\theta'') \sum_{s \in S} \pi(s|\theta'')
		\\
		&= p(\theta'') .
	\end{align*}

	To show that \ref{split:bary} implies \ref{split:induc}, fix $p \in \interior(\Delta(\Theta))$ and $\tau \in \Delta^0(\Delta(\Theta))$, and suppose that $\int q \tau( \dd q ) = p$. Let $S \coloneqq \supp(\tau)$, and define $\pi : S \times \Theta \to [0,1]$ by
	%
	\begin{equation*}
		\pi(q|\theta)
		\coloneqq q(\theta) \tau(q) / p(\theta)
		\quad \text{for all $\theta \in \Theta$ and $q \in S$.}
	\end{equation*}
	%
	Note that $S$ is finite and that
	%
	\begin{equation*}
		\sum_{q \in S} \pi(q|\theta)
		= \frac{1}{p(\theta)} \sum_{q \in \supp(\tau)} q(\theta) \tau(q)
		= \frac{1}{p(\theta)} \int q \tau( \dd q )
		= 1 .
	\end{equation*}
	%
	Hence $\langle S, \pi \rangle $ is a signal structure. And for each $q \in S$, we have
	%
	\begin{multline*}
		p_{\langle S, \pi \rangle }(\theta|q)
		= \frac{p(\theta) \pi(q|\theta)}{\sum_{\theta' \in \Theta} p(\theta') \pi(q|\theta')} 
		\\
		= \frac{p(\theta) q(\theta) \tau(q) / p(\theta)}{\sum_{\theta' \in \Theta} p(\theta') q(\theta') \tau(q) / p(\theta')} 
		= q(\theta) 
		\quad \text{for every $\theta \in \Theta$.}
	\end{multline*}
	%
	Hence $p$ and $\langle S, \pi \rangle $ together induce $\tau$: for each $q \in \Delta(\Theta)$,
	%
	\begin{align*}
		\sum_{\substack{(\theta,q') \in \Theta \times S :\\ p_{\langle S, \pi \rangle }(\theta|q') = q}} p(\theta) \pi(q'|\theta)
		&= \sum_{\substack{(\theta,q') \in \Theta \times S :\\ q' = q}} p(\theta) \pi(q'|\theta)
		\\
		&= \sum_{\theta \in \Theta} p(\theta) \pi(q|\theta)
		\\
		&= \sum_{\theta \in \Theta} p(\theta) q(\theta) \tau(q) / p(\theta)
		= \tau(q) . \qedhere
	\end{align*}
	%
\end{proof}

The signal structure $\langle S, \pi \rangle $ constructed in the proof that \ref{split:bary} implies \ref{split:induc} is one in which each signal realisation $s=q$ is itself a belief, and (given prior belief $p$) the Bayesian posterior belief after observing signal $s=q$ is this selfsame belief $q$. We could call this an `obedient direct signal structure': it simply tells the decision-maker what to believe, and is constructed in such a way that the decision-maker always believes what she is told to believe.

\begin{remark}
	%
	\label{remark:split_infinite}
	
	The \hyperref[lemma:split]{splitting lemma} can of course be extended from the finite-support posterior-belief distributions $\Delta^0(\Delta(X))$ to the set $\Delta(\Delta(X))$ of all probability measures on (the Borel $\sigma$-algebra on) $\Delta(X)$. This requires relaxing the definition of a signal structure $\langle S, \pi \rangle $ by allowing $S$ to have infinitely many elements.
	%
\end{remark}

The importance of the \hyperref[lemma:split]{splitting lemma} comes from the fact that in standard economic models (without things like framing effects), all that matters about a signal structure $\langle S, \pi \rangle $ is which distribution $\tau$ of posterior beliefs it (together with the prior belief $p$) induces. Concretely, a typical model features a (subjective-)expected-utility decision-maker choosing an action $a$ from a non-empty finite set $A$ to maximise $\sum_{\theta \in \Theta} u(a,\theta) q(\theta)$, where $q \in \interior(\Delta(\Theta))$ is her belief.%
	\footnote{In the language and notation of the Savage framework (\cref{ch_ambi}), the set of payoff-relevant consequences as $Z \coloneqq A \times \Theta$, the risk attitude $u$ is a map $Z \to \R$, and each action $a \in A$ corresponds to the Savage act $f_a : \Theta \to Z$ given by $f_a(\theta) \coloneqq (a,\theta)$ for each $\theta \in \Theta$.}
Then letting $a : \Delta(\Theta) \to \R$ be an optimal decision rule,%
	\footnote{That is, one which satisfies $a(q) \in \argmax_{a \in A} \sum_{\theta \in \Theta} u(a,\theta) q(\theta)$ for each $q \in \Delta(\Theta)$.}
the decision-maker's expected payoff is $U(q) \coloneqq \sum_{\theta \in \Theta} u(a(q),\theta) q(\theta)$, i.e. it is a function of the belief $q \in \Delta(\Theta)$ alone. Her ex-ante expected utility then depends only on the distribution $\tau \in \Delta^0(\Delta(\Theta))$ of posterior beliefs: it is $\mathcal{U}(\tau) \coloneqq \int U \dd \tau$.

The same goes for the payoff of any other party, e.g. other decision-makers or a `principal': if their payoff function is $v : \Theta \times A \to \R$, then their interim expected payoff depends only on the posterior belief $q \in \Delta(\Theta)$, via $V(q) \coloneqq \sum_{\theta \in \Theta} v(a(q),\theta) q(\theta)$ for each $q \in \Delta(\Theta)$, so their ex-ante expected payoff depends only on the distribution $\tau$ of posterior beliefs, via $\mathcal{V}(\tau) \coloneqq \int V \dd \tau$. Similar remarks apply if there are multiple decision-makers taking actions; in that case, $A$ is a set of action profiles, and $a(q)$ is an equilibrium given belief $q \in \Delta(\Theta)$ (where in case of multiple equilibria, some selection is made).

This was all just to say that all what matters about a signal structure $\langle S, \pi \rangle $ is which distribution $\tau$ of posterior beliefs it (together with the prior belief $p$) induces. This is useful because working with distributions of posterior beliefs is much more tractable than signal structures. This tractability is what has made the literatures on information design and on costly information acquisition (a.k.a. `rational inattention') take off, for example. However, if we are to move from working with signal structures to working with distributions of beliefs, we must first identify which are the feasible distributions, i.e. which distributions are actually induced by some signal structure (and the prior belief $p$). The \hyperref[lemma:split]{splitting lemma} answers this question, and the answer is furthermore very simple: property~\ref{split:bary} describes a constraint that is mathematically very tractable.



%%%%%%%%%%%%%%%%%%%%%%%%%%%%%%%%%%%
%%%%%%%%%%%%%%%%%%%%%%%%%%%%%%%%%%%
\section{Blackwell's theorem}
\label{info:blackwell}
%%%%%%%%%%%%%%%%%%%%%%%%%%%%%%%%%%%
%%%%%%%%%%%%%%%%%%%%%%%%%%%%%%%%%%%

In this section, we study the (instrumental) value of information. In particular, we imagine that the decision-maker must choose from a non-empty finite set $A$ of actions, and that her payoff is given by $u : A \times \Theta \to \R$. She has (subjective-)expected-utility preferences, so her interim expected payoff as a function of her posterior belief $q \in \Delta(\Theta)$ is $U(q) \coloneqq \max_{a \in A} \sum_{\theta \in \Theta} u(a,\theta) q(\theta)$, and thus her ex-ante expected payoff as a function of the distribution $\tau$ of posterior beliefs is $\int U \dd \tau$. We can more cumbersomely write this out directly in terms of the decision-maker's signal structure $\langle S, \pi \rangle $ and prior belief $p \in \interior(\Delta(\Theta))$, as follows:
%
\begin{equation*}
	\mathcal{U}(\langle S, \pi \rangle , p)
	\coloneqq \sum_{s \in S} \left( \max_{a \in A} \sum_{\theta \in \Theta} u(a,\theta) p_{\langle S, \pi \rangle }(\theta|s) \right)
	\left( \sum_{\theta \in \Theta} \pi(s|\theta) p(\theta) \right) .
\end{equation*}

Our language so far has described the decision-maker's problem via the pair $(A,u)$; in other words, we have been distinguishing between actions $a \in A$ and their payoff consequences, captured by the vector $( u(a,\theta) )_{\theta \in \Theta} \in \R^\Theta$ of real numbers that the payoffs of action $a \in A$ in each state. (We could write $\R^{\abs*{\Theta}}$ instead of $\R^\Theta$, if you prefer.) All that matters about an action is its payoff vector. For the rest of this section, we will therefore identify each action $a$ with its payoff vector $b = (u(a,\theta))_{\theta \in \Theta} \in \R^\Theta$. The decision-maker's problem is then to choose an `action' $b \in B$ from a non-empty finite set $B \subseteq \R^\Theta$, with the payoff in state $\theta$ of choosing action $b \in B$ being $b(\theta) \in \R$ (the $\theta$th entry of the vector $b$).%
	\footnote{This parsimonious formalism is due to \textcite{Blackwell1951,Blackwell1953}. So if you like mnemonics, think of `$b$' and `$B$' as standing for `Blackwell' (just as `$a$' and `$A$' stand for `action').}

In this more compact language, the decision-maker's interim payoff as a function of her posterior belief $q \in \Delta(\Theta)$ is
%
\begin{equation*}
	V_B(q) \coloneqq \max_{b \in B} \sum_{\theta \in \Theta} b(\theta) q(\theta) ,
	\quad \text{or more parsimoniously,} \quad
	V_B(q) = \max_{b \in B} b \cdot q .
\end{equation*}
%
Her ex-ante expected payoff as a function of the distribution $\tau \in \Delta^0(\Delta(\Theta))$ of posterior beliefs is then $\int V_B \dd \tau$. In terms of the signal structure $\langle S, \pi \rangle $, the ex-ante expected payoff is
%
\begin{equation*}
	\mathcal{V}_B(\langle S, \pi \rangle , p)
	\coloneqq \sum_{s \in S} \left( \max_{b \in B} \sum_{\theta \in \Theta} b(\theta) p_{\langle S, \pi \rangle }(\theta|s) \right)
	\left( \sum_{\theta \in \Theta} \pi(s|\theta) p(\theta) \right) .
\end{equation*}


Although I have talked as if the decision problem $B$ were given, we shall in fact consider all possible $B$s. The set of all possible decision problems is, recall, the set of all non-empty finite subsets of $\R^\Theta$. We call a signal structure \emph{(Blackwell) more informative} than another iff the former yields a higher expected payoff in \emph{every} decision problem:

\begin{definition}[\cite{Blackwell1951,Blackwell1953}]
	%
	\label{definition:moreinfo}
	%
	Let $\Theta$ be non-empty and finite, let $\langle S, \pi \rangle $ and $\langle S', \pi' \rangle $ be signal structures, and fix a (prior) belief $p \in \interior(\Delta(\Theta))$. Write $\tau,\tau' \in \Delta^0(\Delta(\Theta))$ for the distributions of posterior beliefs induced by $\langle S, \pi \rangle $ and $p$ and by $\langle S', \pi' \rangle $ and $p$, respectively. We say that $\langle S, \pi \rangle $ is \emph{Blackwell less informative than} $\langle S', \pi' \rangle $ (given $p$) if and only $\int V_B \dd \tau \leq \int V_B \dd \tau'$ for every non-empty finite $B \subseteq \R^\Theta$.
	%
\end{definition}

This is a comparative notion of the \emph{instrumental} value of information: the value of information lies entirely in its capacity to improve decision-making.

\begin{remark}
	%
	\label{remark:blackwell_priors}
	%
	\Cref{definition:moreinfo} compares signal structures while holding fixed an (arbitrary) prior belief $p \in \interior(\Delta(\Theta))$. One could alternatively define `Blackwell less informative than' in a more demanding way, by requiring that $\int V_B \dd \tau_p \leq \int V_B \dd \tau'_p$ for every non-empty finite $B \subseteq \R^\Theta$ \emph{and every (prior) belief $p \in \Delta(\Theta)$}, where $\tau_p$ ($\tau_p'$) denotes the distribution of posterior beliefs induced by $\langle S, \pi \rangle $ and $p$ ($\langle S', \pi' \rangle $ and $p$). This stronger definition is in fact equivalent to \Cref{definition:moreinfo}; this follows from \hyperref[theorem:blackwell]{Blackwell's theorem} below.
	%
\end{remark}

\begin{definition}[\cite{Blackwell1951,Blackwell1953}]
	%
	\label{definition:garbling}
	%
	Let $\Theta$ be non-empty and finite, and let $\langle S, \pi \rangle $ and $\langle S', \pi' \rangle $ be signal structures. A \emph{garbling kernel from $\langle S', \pi' \rangle $ to $\langle S, \pi \rangle $} is a map $g : S \times S' \to [0,1]$ satisfying $\sum_{s \in S} g(s|s') = 1$ for each $s' \in S'$ such that $\pi(s|\theta) = \sum_{s' \in S} g(s|s') \pi'(s'|\theta)$ for each $\theta \in \Theta$ and $s \in S$. We say that $\langle S, \pi \rangle $ is a \emph{garbling} of $\langle S', \pi' \rangle $ if and only if there exists a garbling kernel from $\langle S', \pi' \rangle $ to $\langle S, \pi \rangle $.
	%
\end{definition}

Garbling is a purely statistical notion of `less informative than', making no reference to decisions or payoffs, quite unlike \Cref{definition:moreinfo}.

Another purely statistical sense in which one signal structure may be `less informative' than another is for the posterior-belief distribution induced by the former signal structure to be less dispersed, meaning that beliefs `move less'. The following is a standard notion of `less dispersed than'.

\begin{definition}[\cite{HardyLittlewoodPolya1934}]
	%
	\label{definition:belief_dispersed}
	%
	Let $\Theta$ be non-empty and finite, and fix $\tau,\tau' \in \Delta^0(\Delta(\Theta))$. $\tau$ is dominated by $\tau'$ in the \emph{convex order,} written $\tau \lesssim_{\text{cvx}} \tau'$, if and only if $\int \phi \dd \tau \leq \int \phi \dd \tau'$ for every continuous convex function $\phi : \Delta(\Theta) \to \R$.
	%
\end{definition}

This is the multi-dimensional version of the convex order discussed in \cref{mone:stoch_high} above; the definition is exactly the same, but the domain $\Delta(\Theta)$ is now a convex subset of $\R^\Theta$ rather than of $\R$. The `embedding' characterisation of the convex order carries over to the multi-dimensional case: $\tau \lesssim_{\text{cvx}} \tau'$ holds if and only if there exists a two-period $\Delta(\Theta)$-valued martingale $(\boldsymbol{q},\boldsymbol{q'})$ such that $\boldsymbol{q} \sim \tau$ and $\boldsymbol{q'} \sim \tau'$. However, the characterisation of the one-dimensional convex order in terms of pointwise inequality of integrated CDFs does \emph{not} extend to the multi-dimensional case.

\begin{namedthm}[Blackwell's theorem {\normalfont \parencite{Blackwell1951,Blackwell1953}}.]
	%
	\label{theorem:blackwell}
	%
	Let $\Theta$ be non-empty and finite, let $\langle S, \pi \rangle $ and $\langle S', \pi' \rangle $ be signal structures, and fix a (prior) belief $p \in \interior(\Delta(\Theta))$. Write $\tau,\tau' \in \Delta^0(\Delta(\Theta))$ for the distributions of posterior beliefs induced by $\langle S, \pi \rangle $ and $p$ and by $\langle S', \pi' \rangle $ and $p$, respectively. The following are equivalent:

	\begin{enumerate}[label=(\alph*)]
	
		\item \label{blackwell:info} $\langle S, \pi \rangle $ is Blackwell less informative than $\langle S', \pi' \rangle $ given $p$.

		\item \label{blackwell:garb} $\langle S, \pi \rangle $ is a garbling of $\langle S', \pi' \rangle $.

		\item \label{blackwell:cvx} $\tau \lesssim_{\text{cvx}} \tau'$.
	
	\end{enumerate}
	%
\end{namedthm}

`Blackwell's theorem proper' is the equivalence of properties~\ref{blackwell:info} and \ref{blackwell:garb}.

\begin{proof}[Proof that properties~\ref{blackwell:info} and \ref{blackwell:cvx} are equivalent]
	%
	Recall that for any non-empty finite $B \subseteq \R^\Theta$, $V_B$ denotes the function $\Delta(\Theta) \to \R$ given by $V_B(q) = \max_{b \in B} b \cdot q$ for each $q \in \Delta(\Theta)$. For any $b \in \R^\Theta$, $q \mapsto b \cdot q$ is affine, hence convex. Since the maximum of convex functions is convex (why?), it follows that for any non-empty finite $B \subseteq \R^\Theta$, $V_B$ is convex. Furthermore, for any non-empty finite $B \subseteq \R^\Theta$, $V_B$ is continuous (why?). Hence if property~\ref{blackwell:cvx} holds, then $\int V_B \dd \tau \leq \int V_B \dd \tau'$ for every non-empty finite $B \subseteq \R^\Theta$, which is to say that property~\ref{blackwell:info} holds.

	For the converse, note that any continuous convex function $\phi : \Delta(\Theta) \to \R$ can be approximated arbitrarily well by $V_B$ for some non-empty finite $B \subseteq \R^\Theta$: precisely, for any continuous convex $\phi : \Delta(\Theta) \to \R$ and any non-empty finite set $Q \subseteq \Delta(\Theta)$, there exists a non-empty finite $B \subseteq \R^\Theta$ such that $V_B=\phi$ on $Q$. In particular, for each $q \in Q$, let $b_q \in \R^\Theta$ be a(n arbitrary) subgradient of $\phi$ at $q$;%
		\footnote{A vector $b \in \R^\Theta$ is a \emph{subgradient} of $\phi$ at $q \in \Delta(X)$ iff the affine function $q' \mapsto b \cdot (q'-q) + \phi(q)$ lies pointwise below $\phi$: that is, iff $b \cdot (q'-q) \leq \phi(q')-\phi(q)$ for every $q' \in \Delta(\Theta)$. $\phi$ admits a subgradient at every $q \in \Delta(X)$ since it is continuous and convex.}
	then $B \coloneqq \{ b_q : q \in Q \}$ has the property that $V_B=\phi$ on $Q$ (convince yourself).

	Now, suppose that property~\ref{blackwell:info} holds, and fix an arbitrary continuous convex function $\phi : \Delta(\Theta) \to \R$; we must show that $\int \phi \dd \tau \leq \int \phi \dd \tau'$. Let $Q \coloneqq \supp(\tau) \cup \supp(\tau')$, and note that it is finite. Hence by the preceding paragraph, there exists a non-empty finite $B \subseteq \R^\Theta$ such that $V_B=\phi$ on $Q$, whence $\int \phi \dd \tau = \int V_B \dd \tau \leq \int V_B \dd \tau' = \int \phi \dd \tau'$.
	%
\end{proof}

\begin{exercise}
	%
	\label{exercise:blackwell_easy}
	%
	Prove that property~\ref{blackwell:garb} implies property~\ref{blackwell:info} in \hyperref[theorem:blackwell]{Blackwell's theorem}. 
	%
\end{exercise}

For a (very nice) proof that \ref{blackwell:info} implies \ref{blackwell:garb}, see section~4.4 in \textcite{Liang2023}, which is based on \textcite{Deoliveira2018}. The key step is an invocation of the separating hyperplane theorem.




%______________________________________________________________________________




%       _                               _ _
%      / \   _ __  _ __   ___ _ __   __| (_) ___ ___  ___
%     / _ \ | '_ \| '_ \ / _ \ '_ \ / _` | |/ __/ _ \/ __|
%    / ___ \| |_) | |_) |  __/ | | | (_| | | (_|  __/\__ \
%   /_/   \_\ .__/| .__/ \___|_| |_|\__,_|_|\___\___||___/
%           |_|   |_|


\begin{appendices}

\crefalias{chapter}{appsec}
\crefalias{section}{appsec}
\crefalias{subsection}{appsec}
\crefalias{subsubsection}{appsec}




%%%%%%%%%%%%%%%%%%%%%%
%%%%%%%%%%%%%%%%%%%%%%
%%%%%%%%%%%%%%%%%%%%%%
\chapter{Mathematical background}
\label{math}
%%%%%%%%%%%%%%%%%%%%%%
%%%%%%%%%%%%%%%%%%%%%%
%%%%%%%%%%%%%%%%%%%%%%

In this appendix chapter, I review some mathematical concepts that are useful for understanding the main text.



%%%%%%%%%%%%%%%%%%%%%%%%%%%%%%%%%%%
%%%%%%%%%%%%%%%%%%%%%%%%%%%%%%%%%%%
\section{Sets and functions}
\label{math:set}
%%%%%%%%%%%%%%%%%%%%%%%%%%%%%%%%%%%
%%%%%%%%%%%%%%%%%%%%%%%%%%%%%%%%%%%

$\N = \{1,2,3,\dots\}$ denotes the natural numbers, $\R$ denotes the real numbers, and for $n \in \N$, $\R^n$ denotes the set of all length-$n$ vectors of real numbers. $\R_+$ denotes the set of non-negative real numbers, $\R_{++}$ those that are strictly positive, i.e. $\R_{++} \coloneqq \R_+ \setminus \{0\}$, and similarly $\R_-$ ($\R_{--}$) denotes the non-positive (strictly negative) reals.

Given non-empty sets $X$, $Y$ and $Z$ and functions $f : X \to Y$ and $g : Y \to Z$, the \emph{composition} of $f$ and $g$ is the function $g \circ f : X \to Z$ defined by $(g \circ f)(x) \coloneqq g(f(x))$ for each $x \in X$. Given non-empty sets $X$, $Y$ and $Z$ such that $X \supseteq Y$ and a function $f : X \to Z$, the \emph{restriction of $f$ to $Y$,} often denoted by `$f|_Y$', is the function $g : Y \to Z$ defined by $g(y) \coloneqq f(y)$ for every $y \in Y$.

For any set $S$, $2^S$ denotes the set of all subsets of $S$ (including the empty set $\varnothing$). For any nested sets $S \subseteq X$, the indicator function $\1_S : X \to \R$ is defined by $\1_S(x) \coloneqq 1$ if $x \in S$ and $\1_S(x) \coloneqq 0$ otherwise.

The \emph{Cartesian product} of two sets $S$ and $R$, denoted $S \times R$, is the set of all pairs $(s,r)$ such that $s \in S$ and $r \in R$. By extension, the Cartesian product of a collection $(S_\iota)_{\iota \in \mathcal{I}}$ of sets (where $\mathcal{I}$ is a non-empty [`index'] set), is defined
%
\begin{equation*}
	\prod_{\iota \in \mathcal{I}} S_\iota
	\coloneqq \left\{
	(s_\iota)_{\iota \in \mathcal{I}} : 
	\text{$s_\iota \in S_\iota$ for each $\iota \in \mathcal{I}$}
	\right\} .
\end{equation*}
%
A set $S$ that may be written as a Cartesian product, i.e. $S = \prod_{\iota \in \mathcal{I}} S_\iota$ where $\abs*{\mathcal{I}} \geq 2$, is called a \emph{product set.}

If $S_\iota = R$ for every $\iota \in \mathcal{I}$ and $\mathcal{I}$ is finite or countable, then we use the simpler notation $R^{\abs*{\mathcal{I}}} \equiv \prod_{\iota \in \mathcal{I}} S_\iota$.
(An example is $\R^n$, the set of real vectors of length $n$; it is exactly the $n$-fold Cartesian product of the real line $\R$.)
A more general notation, which is used also when $\mathcal{I}$ is uncountable, is $R^\mathcal{I} \equiv \prod_{\iota \in \mathcal{I}} S_\iota$.%
	\footnote{This explains why we write `$2^S$' for the set of all subsets of a set $S$. Any subset $R \subseteq S$ may be identified with the \emph{inclusion map} $f : S \to \{\text{in},\text{out}\}$ where $f(s)=\text{in}$ iff $s \in R$, for each $s \in S$. Hence the set of all subsets of $S$ may be identified with the set of all such inclusion maps, i.e. all maps $S \to \{\text{in},\text{out}\}$; this is the set $\{\text{in},\text{out}\}^S$. Of course what matters about the set $\{\text{in},\text{out}\}$ is not the labels `in' and `out', but merely the fact that the set has two elements; so we shorten `$\{\text{in},\text{out}\}^S$' to `$2^S$'.}



%%%%%%%%%%%%%%%%%%%%%%%%%%%%%%%%%%%
%%%%%%%%%%%%%%%%%%%%%%%%%%%%%%%%%%%
\section{Proofs}
\label{math:pf}
%%%%%%%%%%%%%%%%%%%%%%%%%%%%%%%%%%%
%%%%%%%%%%%%%%%%%%%%%%%%%%%%%%%%%%%

`Iff' is shorthand for `if and only if'. `$x \coloneqq y$' means `I hereby define $x$: it is equal to $y$'.

For any entailment claim, i.e. a claim of the form `$A$ implies $B$', the \emph{contra-positive} claim is `{``}not $B$'' implies ``not $A${''}'. Every entailment claim is logically equivalent to its contra-positive. It is common, when wishing to prove a claim, to instead prove its contra-positive.

Any claim that is implied by a collection of true claims must itself be true. Thus if we can show that a certain collection of claims entails a falsehood (a claim which \emph{contradicts} things we know to be true, such as the fact that $2+2=4$ or that $\R$ is uncountable), then we may conclude that at least one claim in the collection is false. This principle is also frequently used in proofs: to show that a claim $A$ is true, I show that the claim `not $A$' together with other known facts (e.g. facts proved earlier, or well-known facts like $2+2=4$) implies a falsehood---more specifically, that they entail a claim which contradicts claims known to be true (like the fact that $\R$ is uncountable). This proof technique is called \emph{proof by contradiction.}

\emph{Mathematical induction} is the following logical principle. Suppose we are interested in a collection $(C_t)_{t=0}^\infty$ of claims; that is, for each $t \in \{0,1,2,\dots\}$, $C_t$ has the form `such-and-such is true when $T=t$'. The principle of mathematical induction is this: in order to prove that $C_t$ is true for every $t \in \{0,1,2,\dots\}$, it suffices to prove both of the following:

\begin{itemize}

	\item The `base case': $C_0$ is true.

	\item The `induction step': for any $t \in \N$, if $C_{t-1}$ is true, then $C_t$ is true.

\end{itemize}
%
In the induction step, the hypothesis `$C_{t-1}$ is true' is called the \emph{induction hypothesis.}
A slightly stronger principle of induction (sometimes called `complete' or `strong' induction) is this: in order to prove that $C_t$ is true for every $t \in \{0,1,2,\dots\}$, it suffices to prove both of the following:

\begin{itemize}

	\item The `base case': $C_0$ is true.

	\item The `(strong) induction step': for any $t \in \N$, if $C_{s-1}$ is true for every $s \in \{1,2,\dots,t\}$, then $C_t$ is true.

\end{itemize}



%%%%%%%%%%%%%%%%%%%%%%%%%%%%%%%%%%%
%%%%%%%%%%%%%%%%%%%%%%%%%%%%%%%%%%%
\section{Measure and integral}
\label{math:meas}
%%%%%%%%%%%%%%%%%%%%%%%%%%%%%%%%%%%
%%%%%%%%%%%%%%%%%%%%%%%%%%%%%%%%%%%

The theory of measure and (Lebesgue) integration are the foundation of modern real analysis, which is in turn the backbone of economic theory.
You do not need to know it to take this course.
But it helps to know some of the basic \emph{language} of measure theory; that's what I'll cover here.

(If you'd like to pursue research in economic theory, then I would advise you to learn basic measure theory.
I taught myself from \textcite{Rosenthal2006}; this book is very accessible, except that chapter~2 is harder than it needs to be, so don't get stuck there! I now prefer the first few chapters of \textcite{Folland1999}, a very beautiful book for first-year graduate students in maths. There are lots of other standard books. Efe Ok has a `measure and probability' manuscript on his website that is specifically aimed at economists.)


Let $X$ be a non-empty set.
A \emph{$\sigma$-algebra} on $X$ is a collection of subsets of $X$ satisfying certain properties.%
	\footnote{The properties are closedness under complement and closedness under countable union.}
If $\mathcal{X}$ is a $\sigma$-algebra, we call $(X,\mathcal{X})$ a \emph{measurable space,} and call the elements of $\mathcal{X}$ the \emph{measurable subsets} of $X$.
Often the $\sigma$-algebra $\mathcal{X}$ is left partly or entirely implicit, by the way.

A \emph{measure} on a measurable space $(X,\mathcal{X})$ is a map $\mu : \mathcal{X} \to [0,\infty]$ that is countably additive:
for any countable collection $A_1,A_2,\dots$ of pairwise disjoint measurable subsets of $X$,
we have $\mu\left( \Union_{n \in \N} A_n \right) = \sum_{n \in \N} \mu(A_n)$.
The triple $(X,\mathcal{X},\mu)$ is called a \emph{measure space.}

A measurable set $A \subseteq X$ is called \emph{$\mu$-null} iff $\mu(A)=0$. If a property holds at every $x \in X$, except possibly for $x$ belonging to a null set $A$, then that property is said to hold \emph{($\mu$-)almost everywhere,} or `($\mu$-)a.e.'.

\begin{example}
	%
	\label{example:lebesgue_measure}
	%
	The most commonly-used measure on $X=\R$ is the \emph{Lebesgue measure,}
	which is the unique measure $\lambda$ with the property that
	$\lambda([a,b]) = b-a$ for all $a<b$.
	That is, it captures the common-sense notion of \emph{length.}

	Analogously, there's Lebesgue measure on $\R^2$, which captures \emph{area,} and Lebesgue measure on $\R^3$, which captures \emph{volume,} and so on.

	The Lebesgue measure is conventionally defined on the \emph{Lebesgue $\sigma$-algebra} (whose elements are called \emph{Lebesgue sets}).
	It is often easier to work with the coarser \emph{Borel $\sigma$-algebra} (or \emph{Borel sets}); this is the smallest $\sigma$-algebra containing every open subset of $\R$.
	%
\end{example}

If $\mu$ has the further property that $\mu(X)=1$, then it is a \emph{probability measure,}
and $(X,\mathcal{X},\mu)$ is a \emph{probability space.}
In the probability context, `almost everywhere' is usually replaced with `almost sure(ly)' or `a.s.'.

Now consider a function $f : X \to Y$, where both $X$ and $Y$ are measurable spaces. For any measurable set $B \subseteq Y$, a measure on $Y$ can tell us how `large' the set $B$ is.
How large, then, is the set $A = \{ x \in X : f(x) \in B \}$ of points $x$ in $X$ that lead to a value $f(x)$ that lives in $B$?
That question only has an answer if $A$ is a measurable set!
We call a function \emph{measurable} if this question has an answer for every set $B$.
Symbolically, $f$ is measurable iff for every measurable subset $B$ of $Y$, $\{ x \in X : f(x) \in B \}$ is a measurable subset of $X$.

In modern analysis, the standard integral is the Lebesgue integral.
This is the integral used almost exclusively in economic theory, including these notes.
The integral of a measurable function $f : X \to \R$ on a measure space $(X,\mathcal{X},\mu)$ is written $\int_X f \dd \mu$.
The integral over a measurable subset $A \subseteq X$ is defined
%
\begin{equation*}
	\int_A f \dd \mu \coloneqq \int_X f \1_A \dd \mu ,
\end{equation*}
%
where $\1_A(x) \coloneqq 1$ for $x \in A$ and $\coloneqq 0$ for $x \notin A$.

On probability spaces, measurable functions are conventionally called \emph{random variables,}
and integrals are called \emph{expectations.}
One writes
%
\begin{equation*}
	\E( f ) \coloneqq \int_X f \dd \mu .
\end{equation*}

If you have taken introductory calculus, then you may be more familiar with the \emph{Riemann integral.}%
	\footnote{The Riemann integral of a function $\R \to \R$ is defined as the limit of the areas of increasingly fine rectangles approximating the area under the graph of the function.}
The Riemann integral has the problem that the integral of some functions simply fail to exist.
The Lebesgue integral extends the Riemann integral:
it allows more functions to be integrated,
while still giving the same result as the Riemann integral whenever the latter exists.

The Lebesgue integral is defined only for functions $f : X \to \R$ that are measurable.
That is a necessary condition, but it is not sufficient:
the existence of the integral requires a further condition.
(If the further condition fails, the definition of the Lebesgue integral yields the expression $\infty - \infty$, which has no meaning; therefore we agree to say that the integral does not exist in such cases.)
The integral of a function may be infinite (equal to $\infty$ or $-\infty$).
A function is called \emph{integrable} iff both (a) its integral exists, \emph{and} (b) its integral is finite.

Whereas the Riemann integral is defined only for functions $f : \R \to \R$,
the Lebesgue integral makes sense for functions $f : X \to \R$
defined on any space $X$ you like, provided it has measurable structure (i.e. is equipped with a $\sigma$-algebra).
This is very useful.



\end{appendices}


%______________________________________________________________________________




%    ____  _ _     _ _                             _
%   | __ )(_) |__ | (_) ___   __ _ _ __ __ _ _ __ | |__  _   _
%   |  _ \| | '_ \| | |/ _ \ / _` | '__/ _` | '_ \| '_ \| | | |
%   | |_) | | |_) | | | (_) | (_| | | | (_| | |_) | | | | |_| |
%   |____/|_|_.__/|_|_|\___/ \__, |_|  \__,_| .__/|_| |_|\__, |
%                            |___/          |_|          |___/


% \pagebreak
\printbibliography[heading=bibintoc]



%______________________________________________________________________________



\end{document}
